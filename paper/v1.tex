\documentclass[prd,amsmath,amssymb,floatfix,superscriptaddress,nofootinbib,twocolumn]{revtex4-1}

\def\be{\begin{equation}}
\def\ee{\end{equation}}
\def\bea{\begin{eqnarray}}
\def\eea{\end{eqnarray}}
\newcommand{\C}{\rm C}
\newcommand{\m}{\rm m}
\newcommand{\rs}{\rm rs}
\newcommand{\LC}{\rm LC}
\newcommand{\ini}{\rm ini}
\newcommand{\vbk}{\vec{\bar{k}}}
\newcommand{\vq}{\vec{q}}
\newcommand{\vrr}{\vec{r}}
\newcommand{\vbkp}{\vec{\bar{k'}}}
\newcommand{\bk}{{\bar{k}}}
\newcommand{\vs}{\nonumber\\}
\newcommand{\Var}{\rm Var}
\newcommand{\vk}{\vec{k}}
\newcommand{\ec}[1]{Eq.~(\ref{eq:#1})}
\newcommand{\eec}[2]{Eqs.~(\ref{eq:#1}) and (\ref{eq:#2})}
\newcommand{\Ec}[1]{(\ref{eq:#1})}
\newcommand{\eql}[1]{\label{eq:#1}}
\newcommand{\rf}[1]{\ref{fig:#1}}
\newcommand{\sfig}[2]{
\includegraphics[width=#2]{../plots/#1}
        }
%\newcommand{\sfigg}[2]{
%\includegraphics[width=0.424\paperwidth]{../plots/#1}
%        }
\newcommand{\sfigg}[2]{
\includegraphics[width=#2]{../plots/#1}
        }
\newcommand{\sfiggg}[2]{
\includegraphics[width=0.4\paperwidth]{../plots/#1}
        }
\newcommand{\sfigr}[2]{
\includegraphics[angle=270,origin=c,width=#2]{#1}
        }
\newcommand{\sfigra}[2]{
\includegraphics[angle=90,origin=c,width=#2]{#1}
        }
\newcommand{\Sfig}[2]{
   \begin{figure}[thbp]
   \begin{center}
    \sfig{../plots/#1.pdf}{\columnwidth}
    \caption{{\small #2}}
    \label{fig:#1}
     \end{center}
   \end{figure}
}
\newcommand{\Sfigg}[2]{
   \begin{figure}[thbp]
    \sfigg{../plots/#1.pdf}{.8\paperwidth}
    \caption{{\small #2}}
    \label{fig:#1}
   \end{figure}
}
\newcommand{\Spng}[2]{
   \begin{figure}[thbp]
   \begin{center}
    \sfigg{../plots/#1.png}{\columnwidth}
    \caption{{\small #2}}
    \label{fig:#1}
     \end{center}
   \end{figure}
}
\usepackage{calligra}
\DeclareMathAlphabet{\mathcalligra}{T1}{calligra}{m}{n}
\DeclareMathAlphabet{\mathpzc}{OT1}{pzc}{m}{it}
\usepackage[utf8]{inputenc}
\usepackage{graphicx}
\usepackage{amssymb}
\usepackage{simplewick}
\usepackage{amsmath}
\usepackage{bm}
\usepackage{color}
\usepackage{enumitem}
\usepackage[linktocpage=true]{hyperref} 
\hypersetup{
    colorlinks=true,       
    linkcolor=red,         
    citecolor=blue,        
    filecolor=magenta,      
    urlcolor=blue           
}
\usepackage[all]{hypcap} 
\definecolor{darkgreen}{cmyk}{0.85,0.1,1.00,0} 
\definecolor{darkorange}{rgb}{1.0,0.2,0.0}
\newcommand{\scott}[1]{{\color{darkgreen} \, [Scott: #1]}}
\newcommand{\peikai}[1]{{\color{blue} #1}}
\newcommand{\prvs}[1]{{\color{magenta} #1}}
\newcommand{\AL}[1]{{\color{magenta} AL: #1}}
\newcommand{\MR}[1]{{\color{blue} MR: #1}}
\newcommand{\RC}[1]{{\color{darkorange} #1}}
\newcommand{\nl}{\\ \indent}
\newcommand\dmh{\delta_{\rm m}^{\rm h}}
\newcommand\hdmh{\hat{\delta}_{\rm m}^{\rm h}}
\begin{document}
\title{Large Scale Structure Reconstruction with Short-Wavelength Modes: \\Halo Bias and Light Cone Formalism}
\author{\large Peikai Li}
\affiliation{Department of Physics, Carnegie Mellon University, Pittsburgh, PA 15213, USA}
\affiliation{McWilliams Center for Cosmology, Carnegie Mellon University, Pittsburgh, PA 15213, USA}
\author{\large Rupert A. C. Croft}
\affiliation{Department of Physics, Carnegie Mellon University, Pittsburgh, PA 15213, USA}
\affiliation{McWilliams Center for Cosmology, Carnegie Mellon University, Pittsburgh, PA 15213, USA}
\author{\large Scott Dodelson}
\affiliation{Department of Physics, Carnegie Mellon University, Pittsburgh, PA 15213, USA}
\affiliation{McWilliams Center for Cosmology, Carnegie Mellon University, Pittsburgh, PA 15213, USA}

\date{\today}
\begin{abstract}
\noindent This is the second paper in a series where we propose a method of indirectly measuring large scale structure using information from small scale perturbations. The idea is to build a quadratic estimator from small scale modes that provides a map of structure on large scales.
We demonstrated in the first paper that the quadratic estimator works well on a dark-matter-only N-body simulation at a snapshot of $z=0$. Here we generalize the theory to the case of a light-cone with halo bias and redshift space distortions taken into consideration. We successfully apply the generalized version of the quadratic estimator to a light cone halo catalog of an N-body simulation of size $\sim5.6\,(h^{-1}\,\rm Gpc)^3$. The most distant point in the light cone is at a redshift of $1.4$, which indicates that we might be able to apply our method to next generation galaxy surveys.
%\noindent This is the second paper in a series where we propose a method of indirectly measuring large scale structure using information of small scale perturbations. The idea is to use two-point off-diagonal terms of density contrast modes to build a quadratic estimator for long-wavelength modes. We demonstrate in the first paper that our quadratic estimator works well on a dark-matter-only N-body simulation of the snapshot $z=0$. Here we generalize our theory to the case of a light-cone with halo bias taken into consideration. We successfully apply our generalized version of the quadratic estimator to a light cone halo catalog of a N-body simulation of size $\sim5.6\,(h^{-1}\,\rm Gpc)^3$. The most distant point in the light cone is at a redshift of $1.4$, which indicates that we might be able to apply our method to next generation galaxy surveys.
%
%\noindent This is the second paper in a series where we propose a method of indirectly measuring large scale structure using information of small scale perturbations. The idea is to use two-point off-diagonal terms of density contrast modes to build a quadratic estimator for long-wavelength modes. We demonstrate in the first paper that our quadratic estimator works well on a dark-matter-only N-body simulation of the snapshot $z=0$. Here we generalize our theory to the case of a light cone with halo bias taken into consideration. We successfully apply our generalized version of the quadratic estimator to a light cone halo catalog of a N-body simulation of size $\sim5.6\,(h^{-1}\,\rm Gpc)^3$. The most distant point in the light cone is at a redshift of $1.4$, which indicates that we might be able to apply our method to next generation galaxy surveys.
\end{abstract}
\maketitle
\section{Introduction} \label{sec1}
\noindent Directly measuring the distribution of matter on large scales is extremely difficult right now as pointed out, e.g., by \cite{Modi:2019hydr}. The attempt to use small scale perturbations to infer large scale information has been frequently discussed in recent years \cite{Baldauf:2011fer}\cite{Jeong:2012foss}\cite{Li:2014ssc}\cite{Zhu:2016tidal}\cite{Barreira:2017res}. In our first work \cite{Li:2020fir}, we proposed a method of indirectly measuring large scale structure using the small scale density contrast. Physically, long- and short-wavelength modes are correlated because small scale modes will grow differently depending on the large scale structure they reside in. This phenomenon leaves a signature that in Fourier space: the two-point statistics of short-wavelength matter density modes will have non-zero off-diagonal terms proportional to long-wavelength modes. This is our starting point for constructing the quadratic estimator for long-wavelength modes. We tested the power of the quadratic estimator using a dark-matter-only catalog from an N-body simulation in the first paper. In this work, we generalize Ref.~\cite{Li:2020fir} to account for three effects that must be accounted for before applying the techniques to upcoming surveys~\cite{LSST:2012ls}\cite{Wfirst:2012jg}\cite{DESI:2019ds}: (i) we observe galaxies, not the dark matter field; (ii) we observe on a light cone not a snapshot; and (iii) peculiar velocities lead to redshift space distortions.
%the previous to a halo catalog within a light cone for observational requirements.

First we need to account for halo bias \cite{Kravtsov:1999hb}\cite{Desjacques:2018rev} . Halo bias is a term relating the halo number density contrast to the matter density contrast. We consider only the leading order of halo bias in this work, similar to the treatments of recent galaxy surveys \cite{Seljak:2004sj}\cite{Chang:2016npo}\cite{Prat:2016xor}. One of the main difficulties with this generalization is that the linear halo bias depends on both redshift and halo mass. We will account for this and construct an estimator of the matter density contrast from halo information.

Observationally a galaxy catalog will be in a light cone \cite{Carroll:1997gr} instead of a snapshot. The typical  treatment is to cut a light cone into several thin redshift bins \cite{Troxel:2017xyo} and analyze the properties within each bin. By doing this we will lose information about the long-wavelength modes along line of sight. Thus in this paper we propose a method of considering all the halos/galaxies in a light cone together, accounting for the redshift difference by multiplying by an extra factor proportional to the inverse of the linear growth function. Since the universe is linear at sufficiently large scales, linear matter density modes at different redshift will still be correlated. With this knowledge we can recover the linear power spectrum on large scales. Using a cubic volume, we construct the quadratic estimator for long-wavelength modes using information from non-zero off-diagonal terms as in Ref.~.\cite{Li:2020fir}.

Another observational effect we need to take into account is redshift space distortion~\cite{Kaiser:1987rsd}. This effect can be straightforwardly included by combining the Kaiser formalism with the perturbative expansion developed in real space.
% similar formalism of perturbative series as real space's.

We begin with our treatment of halo bias in a snapshot by constructing the approximate form of the matter perturbation field using the information from a halo catalog. We then deal with the matter density contrast in a light cone and then build the quadratic estimator, finally adding in redshift space distortions. Finally we apply the estimator to different N-body simulations and successfully extract large scale modes accounting for all three effects. 

\section{Halo Bias} \label{sec2}
\noindent We first review the construction of a quadratic estimator of a dark-matter-only catalog \cite{Li:2020fir} before moving to a halo catalog. Starting from the perturbative series of the matter density contrast in Fourier space \cite{Jain:1994sop}\cite{Bernardeau:2002rev}:
\bea
&&\delta_{\rm m}(\vk;a)=\delta^{(1)}_{\rm m}(\vk;a)+\delta^{(2)}_{\rm m}(\vk;a)+\cdots\vs
&=&\frac{D_1(a)}{D_{\rm ini}}\delta^{(1)}_{\rm m}(\vk;a_{\ini})+\bigg[\frac{D_1(a)}{D_{\rm ini}}\bigg]^2\delta^{(2)}_{\rm m}(\vk;a_{\ini})+\cdots \eql{tayl}
\eea 
where ``m" stands for matter, the superscript $i=1,2,\cdots$ corresponds to the $i$-th order term of the expansion, $D_1$ is the linear growth function while $D_{\ini} = D_1(a=a_{\ini})$ is the value of $D_1$ at some initial time $a_{\rm ini}$, and $\delta_{\rm m}(\vk;a)$ is the full Fourier space matter density contrast in a snapshot when the scale factor was equal to $a$. Thus, $\delta_{\rm m}^{(1)}$ is the linear density contrast, so that the first term on the right in the first line above evolves as shown by the first term on the right in the second line.

When evaluating the two-point function of the full density contrast, cross-terms appear. For example, $\langle \delta_{\rm m}^{(1)}(\vk;a) \delta_{\rm m}^{(2)}(\vk';a) \rangle$ is proportional to $\delta^{(1)}_{\rm m}(\vk+\vk',a)$ if both $\vk$ and $\vk'$ correspond to short wavelengths but their sum is small (long wavelength). Explicitly, keeping terms up to second order, 
\be
\langle {\delta}_{\rm m}(\vec{k}_s;a){\delta}_{\rm m}(\vec{k}_s';a) \rangle =f(\vec{k}_s,\vec{k}_s'){\delta}^{(1)}_{\rm m}(\vec{k}_l;a). \eql{2pt}
\ee 
Here $\vk_s$ and $\vk_s'$ are two short-wavelength modes and $\vk_l$ is a long-wavelength mode ($\vk_s,\vk_s' \gg \vk_l$). They satisfy the squeezed-limit condition $\vk_s+\vk_s'=\vk_l$ and $f$ is given by:
\bea
f(\vec{k}_s,\vec{k}_s')&=&2F_2(-\vec{k}_s,\vec{k}_s+\vec{k}_s')P_{\rm m,lin}(k_s;a)\vs
&+&2F_2(-\vec{k}_s',\vec{k}_s+\vec{k}_s')P_{\rm m,lin}(k_s';a)       .
\eea 
Here $P_{\rm m,lin}$ is the linear matter power spectrum and $F_{2}$ is a function particularly insensitive to the choice of cosmological parameters in a dark-energy-dominated universe \cite{Takahashi:2008to}:
\be
F_{2}(\vk_1,\vk_2)=\frac{5}{7}+\frac{2}{7}\frac{(\vk_1\cdot \vk_2)^2}{k_1^2 k_2^2}+\frac{\vk_1\cdot \vk_2}{2k_1k_2}\bigg[\frac{k_1}{k_2}+\frac{k_2}{k_1}\bigg].\eql{f2}
\ee
\ec{2pt} indicates that we can estimate the long-wavelength modes using small scale information with the following minimum variance quadratic estimator:
\begin{eqnarray}
\hat{\delta}^{(1)}_{\rm m}(\vec{k}_l;a)=A(\vec{k}_l)\int \frac{d^3 \vec{k}_s}{(2\pi)^3} g(\vec{k}_s,\vec{k}_s'){\delta}_{\rm m}(\vec{k}_s;a){\delta}_{\rm m}(\vec{k}_s';a)\vs \eql{quadest}
\end{eqnarray} 
with $\vk_s'=\vk_l-\vk_s$. The normalization factor $A$ is defined by requiring that $\langle \hat{\delta}_{\rm m}^{(1)}(\vec{k}_l;a) \rangle={\delta}_{\rm m}^{(1)}(\vec{k}_l;a)$, and the weighting function $g$ is obtained by minimizing the noise. They can be expressed as:
\begin{eqnarray}
A(\vec{k}_l)&=&\bigg[\int \frac{d^3 \vec{k}_s}{(2\pi)^3} g(\vec{k}_s,\vec{k}_s')f(\vec{k}_s,\vec{k}_s')  \bigg]^{-1} \eql{a} \vs
g(\vec{k}_{s},\vec{k}_{s}')&=&\frac{f(\vec{k}_{s},\vec{k}_{s}')}{2P_{\rm m,nl}(k_{s})P_{\rm m,nl}(k_{s}')}
\end{eqnarray}
where $P_{\rm m,nl}$ is the nonlinear matter power spectrum. With this choice of the weighting function $g$, the noise on the estimator $N(\vk_l)=A(\vk_l)$ if non-gaussian terms in the four-point function are neglected. Therefore, the projected detectability of a power spectrum measurement using this quadratic estimator can be written as:
\be
\frac{1}{\sigma^{2}(k_l)}=\frac{V k_l^2 \Delta k }{(2\pi)^2}\bigg[\frac{P_{\rm m,lin}(k_l)}{P_{\rm m,lin}(k_l)+A(k_l) }\bigg]^2 \eql{error},
\ee
where $V$ is the volume of a survey and $\Delta k$ is the width of long-wavelength mode bins. We also take advantage of the fact that $A(\vk_l)$ does not depend on the direction of the long mode $\vk_l$.

So far we have reviewed the formalism of the large scale quadratic estimator in a dark-matter-only catalog. Now we generalize to a halo catalog.

Ignoring higher order halo bias, the number density contrast of halos can be related to the matter density contrast at a fixed time $a$ as \cite{Desjacques:2018rev}:
\be 
\delta_{\rm h}(\vrr;a) \equiv \frac{n_{\rm h}(\vrr;a)-\bar{n}_{\rm h}(a)}{\bar{n}_{\rm h}(a)}=b_{1}\delta_{\rm m}(\vrr;a) \eql{hdc}
\ee 
here ``$\rm h$" denotes halo, $n_{\rm h}$ is the halo number density field at a given position, $\bar{n}_{\rm h}$ is the mean number density of halos and $b_{1}$ is the linear bias parameter relating halo and matter density contrasts. The linear bias $b_{1}$ is a function of halo mass and position and has been well approximated using analytical expressions \cite{Seljak:2004SW}\cite{Bhattacharya:2011MF}. We will use the Tinker halo bias function throughout this paper \cite{Tinker:2010Tinker}. Since the linear bias $b_1$ depends not only on halo mass but also halo redshift, it will be more challenging for us to use $\delta_{\rm h}$ than $\delta_{\rm m}$ within a light cone unless we use a very narrow mass bin. Therefore we choose to build an approximate matter density contrast out of halo positions in a snapshot following \cite{Pervical:2007GPS}. To distinguish this from the standard matter contrast, we add an $h$ to the subscript. Then,
\be 
\dmh(\vrr;a) \equiv \frac{\eta_{\rm h}(\vrr;a)-\bar{\eta}_{\rm h}(a)}{\bar{n}_{\rm h}(a)} \eql{hm}
\ee 
where $\eta_{\rm h}(\vrr;a)$ is the usual halo density field with each halo weighted by the inverse of its own halo bias, and $\bar{\eta}_{\rm h}(a)$ is its mean value. They can be explicitly written as:
\bea
\eta_{\rm h}(\vrr;a)&=&\sum_{i=1}^{N_{\rm h}}\frac{\delta_{\rm D}(\vrr-\vrr_{i})}{b_{1}(M_{i};a_{i})}\bigg|_{a_{i}=a} \eql{nob}\\
n_{\rm h}(\vrr;a)&=&\sum_{i=1}^{N_{\rm h}}\delta_{\rm D}(\vrr-\vrr_{i})\bigg|_{a}
\eea
where $\delta_{\rm D}$ is the Dirac delta function, $N_{\rm h}$ is the total number of halos and $a_{i}$, $M_{i}$ and $\vrr_{i}$ are the cosmological scale factor, mass and position of the $i$-th halo, respectively. Since we consider only a fixed time in this section, we can use the simplification $b_{1}(M_{i},a_{i})=b_{1}(M_i)$. We can use the Cloud-in-Cell (CIC) or the Triangular Shaped Cloud (TSC) scheme \cite{Sefusatti:2015CIC} to smooth the sharply peaked Dirac delta function numerically\footnote{We use the TSC scheme in this work.}.

\Sfig{hm}{$P_{\rm m}^{\rm h}(k)$ is the approximate matter power spectrum using halo information \ec{hm}.}
\scott{As I mentioned in the email, it might be nice to demonstrate how well this estimator does.} 

We consider \ec{hm} as the general way of estimating the matter density using the information of halos/galaxies only, since in real surveys the bias can be expressed as a function of galaxy luminosity and color \cite{Cresswell:2008Col}. We will not use the halo density contrast $\delta_{\rm h}$ directly, instead we will use $\dmh$ defined in \ec{hm} as the approximate matter density contrast during the calculation of the quadratic estimator \ec{quadest} both in a snapshot and in a light cone.

\section{Demonstration with N-body simulation I} \label{sec3}
\noindent We test the power of the quadratic estimator using a halo catalog from a cosmological N-body simulation. We use the $z=0$ snapshot from BigMDPL, one of the Multi-Dark cosmological simulations \cite{Klypin:2014nov}. This simulation used a flat $\Lambda$CDM model with Planck Collaboration XVI (2014) \cite{Planck:2014cos} cosmological parameters. Halos were found in this catalog using the \textit{Rockstar} code \cite{Behroozi:2013Rock}. We will focus on halos with masses between $2.2 \times 10^{12}h^{-1}M_{\odot}<M < 10^{14}h^{-1}M_{\odot}$ for simplicity (although we have found similar results in other mass bins).

We use the approximate density contrast defined in \ec{hm} and compute its power spectrum; this is equal to the matter power spectrum \scott{can we show both?}. Thus the projected detectability will be the same as our last work, shown in Fig.~\rf{SN_BIGMDPL} \scott{Not sure what the figure is supposed to show; there is one computed power spectrum; does that come from the quadratic estimator on $\dmh$ or $\delta_m$?}.

\Sfig{SN_BIGMDPL}{Long-wavelength power spectrum using the BigMDPL halo catalog, and its error from \ec{error} which can be expressed as $P_{\rm m, lin}(k_l)\sigma(k_l)$. The box size of the survey is be $L=2.5\, h^{-1}\rm\, Gpc$ thus volume $V=L^3$ and width $\Delta k = 2\pi/L$. The integration range for $\vk_s$ is from $0.03 \,h \,\rm Mpc^{-1}$ to $0.22\,h\,\rm Mpc^{-1}$.\scott{Is this using halos?}\peikai{yes}}

We also transform the true Fourier modes $\dmh(\vk_l)$ and the long-wavelength modes estimated using the quadratic estimator, $\hdmh(\vk_l)$, back into real space and compare them in Fig.~\rf{real_snap}. We can see that this quadratic estimator is able to successfully recapture large scale over/under densities. 
 
\Sfig{real_snap}{Comparison of the true density field of the BigMDPL halo catalog within mass bin $2.2 \times 10^{12}h^{-1}M_{\odot}<M < 10^{14}h^{-1}M_{\odot}$ ($\dmh(\vrr)$ computed using the directly measured large scale modes, top row) and the density field from the quadratic estimator ($\hdmh(\vrr)$, middle row). The bottom row shows their difference. Each panel represents a slice through the simulation volume, $2500\,h^{-1}\rm \, Mpc$ wide, and one cell $\sim 357\,h^{-1}\rm \, Mpc$ thick. The integration range of $\vk_{s}$ is from $0.03\, h \rm \, Mpc^{-1}$ to $0.22\, h \rm \, Mpc^{-1}$.}

\section{Light Cone Formalism} \label{sec4}

The difficulty with applying this formalism to real data is that we are able to observe a galaxy a distance $r$ from us as it was when the scale factor was equal to $a(r)$. Here $a=1$ when $r=0$ and decreases as the distance from us increases. Therefore, we do not have access to the full $\dmh(\vec r; a)$ for all values of $\vec r$, only for that thin shell from which light emitted when the scale factor was equal to $a$ would have reached us by now.

\newcommand\dlc{d^{\LC}_{\rm m,h}}
\newcommand\dolc{d^{\LC, 1}_{\rm m,h}}
\newcommand\dtlc{d^{\LC, 2}_{\rm m,h}}
To deal with this, we first define a quantity that can be observed: 
\be
\dlc(\vrr) \equiv \dmh(\vrr;a(r)) \frac{D_{\ini}}{D_{1}(a(r))} \eql{LC}
\ee
where ``$\LC$" stands for ``light cone".  Note that -- even apart from the ratio of growth factors -- $\dlc$ is very different than $\dmh(\vrr, a)$. The latter is defined at a fixed value of the scale factor and requires knowledge of the density at all positions at that value of $a$, knowledge that we cannot obtain. The former, $\dlc$, is the density at the position $\vrr$ at the time at which we observe it $a(|\vrr|)$.
%
%Thus $a(\vrr)=a(r)$ will not depend on the direction of position $\vrr$. The position $\vrr$ runs over the whole light cone. 
We multiply this observable by the ratio of growth factors so that the large scale mode amplitudes at different times will share the same normalization.

%The field $\delta_{\rm m}(\vrr;a(r))$ is the matter density contrast at position $\vrr$ of the snapshot with scale factor $a(r)$. It can be approximated using \ec{hm} when we consider a galaxy catalog. Generally speaking, the matter density contrast $\delta_{\rm m}(\vrr\, ';a(r))$ at an arbitrary position $\vrr\, '$ is unmeasurable since observationally we cannot have direct access to a specific snapshot. While on the shell $|\vrr\, '|=|\vrr|$ of this snapshot, we have:
%\bea 
%\delta_{\rm m}(\vrr\, ';a(r))= \frac{\rho_{\rm m}(\vrr\, ';a(r'))-\bar{\rho}_{\rm m}(a(r))}{\bar{\rho}_{\rm m}(a(r))} \eql{dmm} \vs
%\eea 
%where $\rho_{\rm m}(\vrr\, ';a(r'))$  is the matter density at position $\vrr\, '$ of snapshot $a(r')$, and $\bar{\rho}_{\rm m}(a(r))$ is the mean matter density of snapshot $a(r)$. In this case $\delta_{\rm m}(\vrr\, ';a(r))$ is an observable since we get to know $\rho_{\rm m}(\vrr\, ';a(r'))$ in cosmic surveys, since $\rho_{\rm m}(\vrr\, ';a(r'))$ is the matter density field at position $\vrr\, '$ of the light cone. And if we cut the light cone into several thin slices we are able to get a interpolation function for each value of $\bar{\rho}_{\rm m}(a(r))$. Thus for each snapshot with scale factor $a(r)$, we can measure its matter density contrast $\delta_{\rm m}(\vrr\, ';a(r))$ on a shell (when $|\vrr\, '|=|\vrr|$). Finally we come to the conclusion that the modified density contrast $d_{\rm m}^{\LC}$ is an observable. The point of this construction \ec{LC} is that we know the properties of matter density contrast in a snapshot very well (e.g. \ec{tayl}). 
And we can take advantage of these properties to further perform our derivation. Inverse Fourier transform of the density modes in a snapshot can be written as:
Again, we still do not have access to the value of the Fourier modes $\delta_{\rm m}(\vk;a(r))$. Luckily they will not show up in our final results thus our formalism can be applied in light cone simulations, and eventually surveys. 

Now consider the Fourier transform of the modified density contrast within the light cone:
\bea
\dlc(\vk) &=&\int_{V} d^3\vrr  \, \dlc(\vrr)  e^{-i\vk \cdot\vrr} \vs
&=&\int_{V} d^3\vrr  \, \dmh(\vrr;a(r)) \frac{D_{\ini}}{D_{1}(a(r))}  e^{-i\vk \cdot\vrr}.
\eql{LCM}
\eea
We can make progress now by writing $\dmh$ in the integrand as
\be 
\dmh(\vrr;a(r)) =\int \frac{d^3 \vk}{(2\pi)^3}e^{i\vk\cdot\vrr}\dmh(\vk;a(r))\eql{feq}
.\ee 
Therefore,
\bea
\dlc(\vk) &=&\int \frac{d^3 \vk'}{(2\pi)^3}\, \int_{V} d^3\vrr \, e^{i(\vk-\vk') \cdot\vrr} \vs
&&\times  \left[ \frac{D_{\ini}}{D_{1}(a(r))}  
\,\dmh(\vk';a(r))\right].
\eea
In the case of the snapshot the term in brackets had no $\vrr$ dependence, so the integral over space led to a Dirac delta function, and then the integral over $\vk'$ to a tautology. Now, though, we have an expression for an observable $\dlc(\vk)$ in terms of a theoretical quantity $\dmh(\vk';a(r))$ that cannot be observed but whose behavior is governed by perturbation theory as outlined in \ec{tayl}. 
Therefore, perturbation theory predicts that 
\be 
\dlc(\vk) = \dolc(\vk)+\dtlc(\vk)+\cdots
\ee 
with each order given by:
\be 
d^{\LC,(\rm i)}_{\rm m}(\vk)=\int_{V} d^3\vrr  \, \int \frac{d^3 \vk'}{(2\pi)^3}e^{i(\vk'-\vk)\cdot\vrr}\delta^{(\rm i)}_{\rm m}(\vk';a(r))\frac{D_{\ini}}{D_{1}(a(r))}  \eql{exp}
\ee 
We can prove \scott{Need to add some steps in here.} that the first order term of $d^{\LC}_{\rm m}$ satisfies the following relation:
\be 
\langle d^{\LC,(\rm 1)}_{\rm m}(\vk) d^{\LC,(\rm 1)}_{\rm m}(\vk') \rangle \simeq (2\pi)^3 \delta_{\rm D}(\vk+\vk')P_{\rm m,ini}(k)\eql{lps}
\ee 
where $P_{\rm m,ini}$ is the linear matter power spectrum at the initial time $a_{\rm ini}$. This relation \ec{lps} tells us that the first order term of the modified light cone matter density contrast $d_{\rm m}^{\LC}(\vk)$ (which corresponds to long-wavelength modes) still characterizes the linear evolution information of the light cone, since redshift difference in the light cone has been cancelled by the extra factor $D_{\ini}/D_{1}(a(r))$ of the integrand. Or equivalently speaking, we have $P_{\rm m}^{\rm LC}(k)\simeq P_{\rm m,ini}(k)$ for long-wavelength modes (when $k\rightarrow 0$), as shown in Fig.~\rf{psLC}. $P_{\rm m}^{\rm LC}(k)$ is the power spectrum of density contrast \ec{LC} defined as the diagonal term of the two-point correlation:
\be 
\langle d_{\rm m}^{\rm LC} (\vk)d_{\rm m}^{\rm LC}(\vk')\rangle \simeq (2\pi)^3 \delta_{\rm D}(\vk+\vk')P_{\rm m}^{\rm LC}(k) \eql{2ptlc}
\ee 
There are also off-diagonal terms in \ec{2ptlc} which contains extremely valuable information, and we will look into them in next section \ref{sec5}.

\Sfig{psLC}{Comparison of the light cone power spectrum $P_{\rm m}^{\LC}(k)$ with the initial power spectrum $P_{\rm m,ini}(k)$, both scaled by $1/D_{\ini}^{2}$. $P_{\rm m, ini}(k)$ is plotted according to theory predicted linear power spectrum, and $P_{\rm m}^{\LC}(k)$ is computed using halo catalog of MICE Grand Challenge light cone simulation \cite{Fosalba:2015MI}\cite{Fosalba:2015MII}. Halo mass bin is chosen to be $[2.2 \times 10^{12}h^{-1}M_{\odot},10^{14}h^{-1}M_{\odot}]$. They match each other pretty well on large scales ($k\lesssim 0.02 \,h\,\rm Mpc^{-1}$) despite tiny systematic error. The error bars shows errors due to limited number of independent modes in each bin.}

\section{Quadratic Estimator} \label{sec5}
Using the expressions of \ec{exp}, we can compute the two-point correlations of two short-wavelength Fourier modes $d^{\LC}_{\rm m}(\vk_s)$ and $d^{\LC}_{\rm m}(\vk_s')$. Again we assume the squeezed limit $\vk_l=\vk_s+\vk_s'$:
\bea 
 &&\langle d^{\LC}_{\rm m}(\vk_s)d^{\LC}_{\rm m}(\vk_s') \rangle|_{\vk_s+\vk_s'=\vk_l}\vs
 =&&\langle d^{\LC,(1)}_{\rm m}(\vk_s)d^{\LC,(2)}_{\rm m}(\vk_s') \rangle+\langle d^{\LC,(2)}_{\rm m}(\vk_s)d^{\LC,(1)}_{\rm m}(\vk_s') \rangle \vs \eql{2p}
\eea 
Substituting \ec{exp} into \ec{2p} and evaluate the first bracket as an example:
\bea 
&&\langle d^{\rm LC,(1)}_{\rm m}(\vk_s)d^{\rm LC,(2)}_{\rm m}(\vk_s') \rangle \vs 
&=& \int_V d^{3}\vrr\int_{V} d^{3} \vrr\, ' \int \frac{d^{3}\vk}{(2\pi)^3}\int \frac{d^{3}\vk'}{(2\pi)^3}\vs
&& e^{i(\vk\cdot\vrr+\vk'\cdot\vrr\, ')-i(\vk_s\cdot\vrr+\vk_s'\cdot\vrr\, ')} \frac{D_{\ini}}{D_{1}(a(r))} \frac{D_{\ini}}{D_{1}(a(r'))}  \vs
&\times &\langle \delta_{\rm m}^{(1)}(\vk;a(r))\delta_{\rm m}^{(2)}(\vk';a(r')) \rangle \eql{od}
\eea 
We have computed $\langle \delta_{\rm m,ini}^{(1)}(\vk)\delta_{\rm m,ini}^{(2)}(\vk') \rangle$ in our previous work \cite{Li:2020fir} (where $\delta_{\rm m,ini}^{(1)}(\vk)\equiv\delta_{\rm m}^{(1)}(\vk;a_{\ini})$ ), the result gives:
\be
\langle \delta_{\rm m,ini}^{(1)}(\vk)\delta_{\rm m,ini}^{(2)}(\vk') \rangle =2F_{2}(-\vk,\vk+\vk')P_{\rm m,ini}(k)\delta_{\rm m,ini}^{(1)}(\vk+\vk') 
\ee
We can use this result to further determine the value of the bracket in \ec{od}:
\bea 
&&\langle \delta_{\rm m}^{(1)}(\vk;a(r))\delta_{\rm m}^{(2)}(\vk';a(r')) \rangle \vs
&=&2\frac{D_{1}(a(r))}{D_{\ini}}\bigg[\frac{D_{1}(a(r'))}{D_{\ini}} \bigg]^{2}\langle \delta_{\rm m,ini}^{(1)}(\vk)\delta_{\rm m,ini}^{(2)}(\vk') \rangle  \vs 
&=&2\bigg[\frac{D_{1}(a(r'))}{D_{\ini}} \bigg]^{2}F_{2}(-\vk,\vk+\vk')P_{\rm m,\ini}(k)\delta_{\rm m}^{(1)}(\vk+\vk';a(r)) \eql{odb} \vs
\eea 
where we make use of the definition of the the linear growth factor $D_1(a(r))$. Plugging \ec{odb} into \ec{od}, the only $\vrr\, '$ dependent integral can be written as (here we simply choose a cube volume):
\be 
\int_{V} d^{3}\vrr\, 'e^{-i(\vk'-\vk_s')\vrr\, '}\frac{D_{1}(a(r'))}{D_{\ini}}  \simeq C\,(2\pi)^3\delta_{\rm D}(\vk'-\vk_s')\eql{dt}
\ee 
since $D_1(a(r'))$ is a slowly varying function. Constant $C$ can be further determined via integrating over $\vk\, '$ on both sides of \ec{dt}, which gives:
\be 
C=\frac{D_{1}(a=1)}{D_{\rm ini}}
\ee 
thus:
\bea 
&& \langle d^{\rm LC,(1)}_{\rm m}(\vk_s)d^{\rm LC,(2)}_{\rm m}(\vk_s') \rangle \vs 
&\simeq &2C\int_{V} d^{3} \vrr\int \frac{d^{3}\vk}{(2\pi)^3}F_2(\vk,-\vk+\vk_s')P_{\rm m,\ini}(|\vk-\vk_s'|) \vs
&&\times \frac{D_{\ini}}{D_{1}(a(r))}e^{-i(\vk-\vk_s-\vk_s')\vrr}\delta_{\rm m}^{(1)}(\vk;a(r)) \vs
&\simeq & 2C\,F_{2}(-\vk_s,\vk_s+\vk_s')P_{\rm m,\ini}(k_s) \vs
&& \times \int_{V} d^{3} \vrr\int \frac{d^{3}\vk}{(2\pi)^3}\frac{D_{\ini}}{D_{1}(a(r))}e^{-i(\vk-\vk_s-\vk_s')\vrr}\delta_{\rm m}^{(1)}(\vk;a(r))\vs
&= &2C\,F_{2}(-\vk_s,\vk_s+\vk_s')P_{\rm m,\ini}(k_s)d_{\rm m}^{\rm LC,(1)}(\vk_s+\vk_s') \eql{appr}
\eea 
where in the first step, we perform a redefinition of the integration dummy variable. And we use an approximation in the second step in order to get the first order term $d_{\rm m}^{\rm LC,(1)}(\vk_s+\vk'_s)$ from the integral. 

Again, we have proved that with this construction in \ec{LC}, we can obtain the long-wavelength modes from the off-diagonal terms of short-wavelength modes:
\be 
\langle d^{\LC}_{\rm m}(\vk_s)d^{\LC}_{\rm m}(\vk_s') \rangle|_{\vk_s+\vk_s'=\vk_l} =\mathpzc{f}(\vec{k}_s,\vec{k}_s'){d}^{\rm LC,(1)}_{\rm m}(\vec{k}_l) \eql{2ptdd}
\ee 
with 
\bea
\mathpzc{f}(\vec{k}_s,\vec{k}_s')&=&2C\,F_2(-\vec{k}_s,\vec{k}_s+\vec{k}_s')P_{\rm m,ini}(k_s)\vs
&+&2C\,F_2(-\vec{k}_s',\vec{k}_s+\vec{k}_s')P_{\rm m,ini}(k_s')       
\eea 

The quadratic estimator can be similarly formed as:
\be
\hat{d}^{\rm LC,(1)}_{\rm m}(\vk_l)=\mathpzc{A}(\vk_l)\int \frac{d^{3}\vk_s}{(2\pi)^3} \mathpzc{g}(\vk_s,\vk_s')d^{\LC}_{\rm m}(\vk_s)d^{\LC}_{\rm m}(\vk_s') \eql{estlc}
\ee 
with $\vk_s' = \vk_l-\vk_s$ and $\mathpzc{g}$ being the weighting function. By requiring that $\langle \hat{d}^{\rm LC,(1)}(\vec{k}_l) \rangle={d}^{\rm LC,(1)}(\vec{k}_l)$ we can similarly determine the normalization function $\mathpzc{A}$:
\begin{eqnarray}
\mathpzc{A}(\vec{k}_l)=\bigg[\int \frac{d^3 \vec{k}_s}{(2\pi)^3} \mathpzc{g}(\vec{k}_s,\vec{k}_s')\mathpzc{f}(\vec{k}_s,\vec{k}_s')  \bigg]^{-1} \eql{a}
\end{eqnarray}
Similar to our last work, by minimizing the noise we get the expression for the weighting function $\mathpzc{g}$:
\begin{eqnarray}
&&\mathpzc{g}(\vec{k}_{s},\vec{k}_{s}')
=\frac{\mathpzc{f}(\vec{k}_{s},\vec{k}_{s}')}{2P^{\rm LC}_{\rm m}(k_{s})P^{\rm LC}_{\rm m}(k_{s}')}\vs
&=&C\frac{F_2(-\vec{k}_s,\vec{k}_s+\vec{k}_s')P_{\rm m,ini}(k_s)+F_2(-\vec{k}_s',\vec{k}_s+\vec{k}_s')P_{\rm m,ini}(k_s')}{P^{\rm LC}_{\rm m}(k_{s})P^{\rm LC}_{\rm m}(k_{s}')}\vs 
\end{eqnarray} 
with this choice of $\mathpzc{g}$ the noise term $\mathpzc{N}$ is identical to the normalization factor $\mathpzc{A}$. And the projected detectability is defined similarly as \ec{error}:
\be
\frac{1}{\sigma^{\rm LC}(k_l)^2}=\frac{V k_l^2 \Delta k }{(2\pi)^2}\bigg[\frac{P_{\rm m,ini}(k_l)}{P_{\rm m,ini}(k_l)+\mathpzc{A}(k_l) }\bigg]^2 \eql{error1},
\ee
%In Fig.~\rf{SN_LC}, we show the projected error of measuring the initial power spectrum using this quadratic estimator \ec{estlc} in a light cone. The difference of the error bar in this plot from the error bar of the cosmic variance ($\mathpzc{A}=0$) is very tiny.

%\Sfig{SN_LC}{Initial power spectrum of a light cone and its error from \ec{error1} which can be expressed as $P_{\rm m, ini}(k_l)\sigma_{\rm LC}(k_l)$. Boxsize of the survey is assumed to be $L=1.773\, h^{-1}\rm\, Gpc$ thus volume $V=L^3$ and width $\Delta k = 2\pi/L$. The integration range for $\vk_s$ is from $0.03 \,h \,\rm Mpc^{-1}$ to $0.22\,h\,\rm Mpc^{-1}$.}
Using the quadratic estimator \ec{estlc} we can use small scale information of the whole light cone (for now we only consider a cube-shaped volume) to infer large scale field of the modified matter density contrast $d_{\rm m}^{\LC}(\vrr)$. According to \ec{lps}, this is equivalent to the large scale linear matter perturbations in the light cone.

\section{Demonstration with N-body simulation II} \label{sec6}
\subsection{Real Space}
\noindent We use the MICE Grand Challenge light cone N-body simulation (MICE-GC) to demonstrate the power of the estimator in a light cone. The catalog contains one octant of the full sky up to $z = 1.4$ (comoving distance $3072\, h^{-1}\, \rm Mpc$) without simulation box repetition. We use the largest cube that can be fit into the octant as the region $V$ of integration in \ec{LCM}, as shown in Fig.~\rf{Cube}. This simulation used a flat $\Lambda$CDM model with cosmological parameters $ \Omega_{\rm m}=0.25$, $\sigma_8 = 0.8$, $n_{\rm s}=0.95$, $\Omega_{\rm b}=0.044$, $\Omega_{\Lambda}=0.75$, $h=0.7$. We also assume $D_{\ini}=1$ throughout this section\footnote{$D_{\rm ini}=1$ is physically ill-defined, because the maximum value of $D_{\rm ini}$ is about $0.997$ for a flat universe with $\Omega_{\rm m}=0.25$. But $D_{\rm ini}$ is simply a normalization constant and the choice of its value does not affect the feature of the final result.}. 

\Sfig{Cube}{Boxsize of the cube is $L=3072/\sqrt{3}\, h^{-1}\, \rm Mpc\sim 1774\, h^{-1}\, \rm Mpc$. Volume is $V=L^3\sim 5.6\,(h^{-1}\,\rm Mpc)^3$. Only one point in the cube can reach the redshift of $1.4$, and $z=0$ is at the origin $O$ of the octant.}

We first extract $1$ in $700$  matter particles' positions in the full the light cone of MICE-GC simulation and then focus on the largest cube within it. This corresponds to $\sim 5.14\times 10^{7}$ particles in total each with $2.9\times 10^{10} h^{-1}M_{\odot}$ particle mass. Thus we have a number density of $3.4\times 10^{-3}$ $\rm particles/(Mpc/h)^3$, which is similar to Dark Energy Survey (DES) \cite{DES:2016DES} full sample of galaxies. We use Nbodykit \cite{Hand:2018nby} to compute the modified matter density contrast field in Fourier space and get the estimated one using \ec{estlc}. Than we transform them back into real space and compare them it Fig.~\rf{real_dm}. The cube with volume $\sim5.6\,(h^{-1}\,\rm Gpc)^3$ is divided into $7^3$ cells. Each row in Fig.~\rf{real_dm} contains 7 panels; each panel is a slice of the cube. We can see that this estimator does an amazing job of extracting the large scale density job - we are able to reproduce almost every cell with large over- or under-densities, and the difference is apparently much smaller than the density field itself.

In Fig.~\rf{real_dm}, the left top cell of each row corresponds to the cell containing the origin ($z=0$) in it; and the right bottom cell corresponds to the farthest cell from the origin ($z\sim 1.4$). We can use Fig.~\rf{cube_dm} to get a better view of this in a light cone, where in each row we still show the usual directly measure field $d_{\rm m}^{\LC}(\vrr)$, estimated field $\hat{d}_{\rm m}^{\LC}(\vrr)$ and their difference $(d_{\rm m}^{\LC}-\hat{d}_{\rm m}^{\LC})(\vrr)$ respectively. We have two columns here, the left column shows cells near the origin and the right column shows cells far from the origin. While all the information in this figure is contained in Fig.~\rf{real_dm}, it still is a more straightforward way of expressing our result in a light cone. 


\Sfig{real_dm}{Comparison of the true density field of the $1$ in $700$ matter particles of the Mice-GC simulation ($\delta^{\rm LC}_{\rm m}(\vrr)$, top row) and the density field from the quadratic estimator ($\hat{\delta}_{\rm m}^{\rm LC}(\vrr)$, middle row). The bottom row shows their difference. Each panel represents a slice through the simulation volume, $1774\,h^{-1}\rm \, Mpc$ wide, and one cell $\sim 253\,h^{-1}\rm \, Mpc$ thick. The integration range of $\vk_{s}$ is from $0.03\, h \rm \, Mpc^{-1}$ to $0.22\, h \rm \, Mpc^{-1}$.}
\Sfig{cube_dm}{Comparison of the true density field and the estimated density field of dark-matter-only catalog in a light cone with the same colorbar as Fig.~\rf{real_dm}. First row shows the modified matter density contrast near the origin (inner surfaces of the cube, $0\leqslant z \lesssim 0.69$); second row shows the modified matter density contrast far from the origin (outer surfaces of the cube, $0.69\lesssim z \lesssim 1.4$).}

Then we consider the halo catalog in the same light cone with halo masses between $2.2 \times 10^{12}h^{-1}M_{\odot}<M < 10^{14}h^{-1}M_{\odot}$ for brevity. We have similar results in other mass bins as well. We use \ec{hm} as the approximated matter density contrast, $\mathpzc{n}_{\rm h}(\vrr)$ can be easily computed using halo positions and the Tinker bias function. We cut the whole light cone (the octant) into several slices and compute the mean density field $\bar{{n}}_{\rm h}$ and $\bar{\mathpzc{n}}_{\rm h}$ of each slice, then get the mean field  $\bar{{n}}_{\rm h}(a(r))$ and $\bar{\mathpzc{n}}_{\rm h}(a(r))$ at each position by interpolation. We compute the directly measured modified matter contrast $d_{\rm m}^{\LC}(\vrr)$ and also the power spectrum in Fig.~\rf{psLC}. The power spectrum perfectly characterizes linear matter power spectrum on large scales. We use the quadratic estimator \ec{estlc} to get the reconstructed modified field and transform them back into real space. Again we have two similar plots Fig.~\rf{real_halo} and Fig.~\rf{cube_halo}. From Fig.~\rf{real_halo} we can see that our quadratic estimator is still able to extract large scale information, especially large over- or under-density cells on first few low-redshift panels. The difference seems larger when we go higher into redshift, and we have the worst performance on the very right panel. We observe the same feature in Fig.~\rf{cube_halo} where the cells near the origin match each other better than cells far from the origin. The main reason for this effect is that nonlinear bias ($b_{2}$, $b_{3}\,\cdots$) becomes more important when we go deeper into the light cone \cite{Lazeyras:2016nbs}.

\Sfig{real_halo}{Comparison of the true density field of the halo number density contrast in mass bin $2.2 \times 10^{12}h^{-1}M_{\odot}<M < 10^{14}h^{-1}M_{\odot}$ of the Mice-GC simulation ($\delta^{\rm LC}_{\rm m}(\vrr)$, top row) and the density field from the quadratic estimator ($\hat{\delta}_{\rm m}^{\rm LC}(\vrr)$, middle row) in the same light cone as Fig.~\rf{real_dm}, and their difference (bottom row). The integration range of $\vk_{s}$ is from $0.03\, h \rm \, Mpc^{-1}$ to $0.35\, h \rm \, Mpc^{-1}$. We use a larger upper limit to slightly reduce the noise term.} 
\Sfig{cube_halo}{Comparison of the true density field and the estimated density field of halo catalog in a light cone with the same colorbar as Fig.~\rf{real_dm} (or Fig.~\rf{real_halo}). First row shows the modified matter density contrast near the origin (inner surfaces of the cube, $0\leqslant z \lesssim 0.69$); second row shows the modified matter density contrast far from the origin (outer surfaces of the cube, $0.69\lesssim z \lesssim 1.4$).}

\subsection{Redshift Space} \label{sec7}
Distribution of halos (and galaxies) is distorted and squashed observationally, when we use their redshift as an indicator of their radial distance due to halos' peculiar velocity. In the plane-parallel approximation, the mapping from real space to redshift space is given by \cite{Bernardeau:2002rev}:
\be 
\vrr_{\rs} = \vrr + \frac{(\vec{u}\cdot \hat{n})\hat{n}}{aH(a)}
\ee 
where $\vrr_{\rs}$ is the redshift space coordinates, ``$\rs$'' stands for ``redshift space'', $\vec{u}$ is the peculiar velocity field, $\hat{n}$ is the direction of the line of sight and $H(a)$ is the Hubble parameter. Similarly, the matter density contrast in the redshift space can be written as a perturbative series:
\bea 
\delta_{\m,\rs}(\vk;a)&=&\sum_{n=1}^{\infty} \int\frac{d^3 \vk_1}{(2\pi)^3} \cdots \int \frac{d^3 \vk_n}{(2\pi)^3} \delta_{\rm D}(\vk-\vk_1 - \cdots -\vk_n) \nonumber \\
&\times & Z_n (\vk_1,\cdots,\vk_n;a)\delta_{\m}^{(1)}(\vk_1;a)\cdots \delta_{\m}^{(1)}(\vk_n;a) \eql{rspt} \vs
\eea
where 
\bea 
&&Z_1(\vk;a)=1+f\mu^2 \\ 
&&Z_2(\vk_1,\vk_2;a)=F_{2}(\vk_1,\vk_2)+f\mu^2 G_2(\vk_1,\vk_2)\vs 
&&\qquad \qquad\quad+\frac{f\mu k}{2}\bigg[ \frac{\mu_1}{k_1}(1+f\mu_2^2)+\frac{\mu_2}{k_2}(1+f\mu_1^2)\bigg]
\eea 
with $\mu \equiv \vk \cdot \hat{n}/k$ and $\mu_{1,2} \equiv \vk_{1,2} \cdot \hat{n}/k_{1,2}$, and,
\bea 
f(a)&\equiv & \frac{d \ln D_{1}(a)}{d \ln a}\\
G_2(\vk_1,\vk_2)&=&\frac{3}{7}+\frac{4}{7}\frac{(\vk_1\cdot \vk_2)^2}{k_1^2 k_2^2}+\frac{\vk_1\cdot \vk_2}{2k_1k_2}\bigg[\frac{k_1}{k_2}+\frac{k_2}{k_1}\bigg]\vs
\eea 
The modified matter density contrast in the redshift space of a light cone can then be expressed as: 
\be 
d^{\LC}_{\rm m,\rs}(\vk)=\int_{V} d^3\vrr  \, \int \frac{d^3 \vk'}{(2\pi)^3}e^{i(\vk'-\vk)\cdot\vrr}\delta_{\rm m,\rs}(\vk';a(r))\frac{D_{\ini}}{D_{1}(a(r))} 
\ee 
One can prove that this definition still satisfies the relation (see appendix \ref{appenda} for a detailed derivation):
\be 
\langle d^{\LC}_{\rm m,\rs }(\vk_s)d^{\LC}_{\rm m,\rs }(\vk_s') \rangle|_{\vk_s+\vk_s'=\vk_l} =\mathpzc{f}_{\rs}(\vec{k}_s,\vec{k}_s'){d}^{\rm LC,(1)}_{\rm m,\rs}(\vec{k}_l) \eql{2ptddrs}
\ee 
with 
\bea
&&\mathpzc{f}_{\rs}(\vec{k}_s,\vec{k}_s')=2C\frac{Z_{1}(\vk_s;a(r\simeq L/2 ))}{Z_{1}(\vk_s+\vk_s';a(r\simeq L/2 ))}\vs
&\times&\bigg[\,Z_2(-\vec{k}_s,\vec{k}_s+\vec{k}_s';a=1)P_{\rm m,ini}(k_s)\vs
&&+\,Z_2(-\vec{k}_s',\vec{k}_s+\vec{k}_s';a=1)P_{\rm m,ini}(k_s')\bigg]  \vs   
\eea 
where $L=3072\, h^{-1}\, \rm Mpc$ when we use MICE-GC simulation which is the depth of the simulation. It comes from the approximation when we take the ratio term out of the integrand. The redshift dependence of the $f$ function is solved similarly as \ec{dt}. 

In Fig.~\rf{SN_RS} we use the halo catalog of MICE-GC simulation to reconstruct the monopole moment of the redshift space matter power spectrum.

\Sfig{SN_RS}{Solid curve is the theoretical curve for the monopole moment of the linear redshift space matter power spectrum from Kaiser formula (detailed discussion in appendix \ref{appenda}). Blue dots and their corresponding error bars are measured using the quadratic estimator in redshift space. The integration range for $\vk_s$ is from $0.03 \,h \,\rm Mpc^{-1}$ to $0.35\,h\,\rm Mpc^{-1}$.}

\section{Conclusion} \label{sec8}

\acknowledgements
We thank Duncan Campbell for resourceful discussions. We also thank Enrique Gaztanaga for providing us with the $1$ in $700$ matter particles' positions of Mice-GC simulation. This work is supported by U.S. Dept. of Energy contract DE-SC0019248 and NSF AST-1909193.
The BigMDPL simulation was performed at LRZ Munich within the PRACE project pr86bu. The CosmoSim database (\url{www.cosmosim.org}) providing the file access is a service by the Leibniz-Institute for Astrophysics Potsdam (AIP).
This work has made use of CosmoHub. CosmoHub has been developed by the Port d'Informació Científica (PIC), maintained through a collaboration of the Institut de Física d'Altes Energies (IFAE) and the Centro de Investigaciones Energéticas, Medioambientales y Tecnológicas (CIEMAT), and was partially funded by the ``Plan Estatal de Investigación Científica y Técnica y de Innovación" program of the Spanish government.
\clearpage

\appendix 

\section{Approximation in \ec{dt}}
We use
\Sfig{approx}{1D demonstration of the approximation made in \ec{dt}. Here $\Delta k=2\pi /L$.}
\section{Proof of \ec{2ptddrs} in Redshift Space} \label{appenda}
One can prove that the leading order term of of $d^{\LC}_{\rm m}(\vk)$ still characterizes the linear matter power spectrum in redshift space:
\bea 
&&\langle d^{\LC,(\rm 1)}_{\rm m}(\vk) d^{\LC,(\rm 1)}_{\rm m}(\vk') \rangle\vs
& = &(2\pi)^3 \delta_{\rm D}(\vk+\vk')(1+f(a=1)\mu^2)^2 P_{\rm m,ini}(k)\vs
&= &(2\pi)^3 \delta_{\rm D}(\vk+\vk')\bigg[ \frac{D_{\ini}}{D_1(a=1)}\bigg]^2 P_{\rm m,rs,lin}(\vk;a=1)\vs \eql{lpsrs}
\eea 
where 
\be 
P_{\rm m,rs,lin}(\vk;a)=(1+f(a)\mu^2)^2 P_{\rm m,lin}(k;a) \eql{kaiser}
\ee 
is the leading order redshift space matter power spectrum first derived by N. Kaiser \cite{Kaiser:1987rsd}. We can decompose its direction-dependence by Legendre polynomials expansion as:
\be 
P_{\rm m,rs,\ell}(k;a)\equiv \frac{2\ell+1}{2}\int_{-1}^{1}d\mu P_{\rm m,rs}(\vk;a) \mathcal{L}_{\ell}(\mu)
\ee 
The monopole ($\ell=0$) and the quadrupole ($\ell=2$) moments have been measured in recent surveys \cite{Gil-Marin:2015sqa}. Now we start our proof of \ec{2ptddrs}, first we have:
\bea 
&&\langle \delta_{\rm m,\rs}^{(1)}(\vk;a(r))\delta_{\rm m,\rs}^{(2)}(\vk';a(r'))\rangle\vs 
&=&Z_{1}(\vk;a(r))\langle \delta_{\rm m}^{(1)}(\vk;a(r))\delta_{\rm m,\rs}^{(2)}(\vk';a(r'))\rangle\vs 
&=&2Z_{1}(\vk;a(r))\bigg[\frac{D_{1}(a(r'))}{D_{\ini}} \bigg]^{2}Z_{2}(-\vk,\vk+\vk';a(r'))\vs
&& \times P_{\rm m,\ini}(k)\delta_{\rm m}^{(1)}(\vk+\vk';a(r))\vs 
&=&\frac{Z_{1}(\vk;a(r))}{Z_{1}(\vk+\vk';a(r))}\bigg\{2\bigg[\frac{D_{1}(a(r'))}{D_{\ini}} \bigg]^{2}Z_{2}(-\vk,\vk+\vk';a(r')) \vs
&& \times P_{\rm m,\ini}(k)\delta_{\rm m,\rs}^{(1)}(\vk+\vk';a(r))\bigg\}
\eea 
where in the step we use the first order expression of $\delta_{\rm m, \rs}$ in \ec{rspt}; and second step can be similarly derived as \ec{odb} with $F_2$ replaced by $Z_2$; and in the last step we use the expression again:
\be 
\delta_{\rm m,\rs}^{(1)}(\vk;a(r))=Z_{1}(\vk;a(r))\delta_{\rm m}^{(1)}(\vk;a(r))
\ee 
in order to get the similar expression as \ec{odb} in the braces. So when we evaluate $\langle d_{\rm m,\rs}^{\LC,(1)}(\vk_s) d_{\rm m,\rs}^{\LC,(2)}(\vk'_s)\rangle$ we get an expression close to \ec{appr}:
\bea 
&& \langle d^{\rm LC,(1)}_{\rm m,\rs}(\vk_s)d^{\rm LC,(2)}_{\rm m,\rs}(\vk_s') \rangle \vs 
&\simeq & 2C\,Z_{2}(-\vk_s,\vk_s+\vk_s';a=1)P_{\rm m,\ini}(k_s) \vs
&& \times \int_{V} d^{3} \vrr\int \frac{d^{3}\vk}{(2\pi)^3}\frac{Z_{1}(\vk-\vk_s';a(r))}{Z_{1}(\vk;a(r))}\frac{D_{\ini}}{D_{1}(a(r))}\vs
&&\qquad e^{-i(\vk-\vk_s-\vk_s')\vrr}\delta_{\rm m}^{(1)}(\vk;a(r))\vs
&\simeq & \frac{Z_{1}(\vk_s;a(r\simeq L/2 ) )}{Z_{1}(\vk_s+\vk_s';a(r\simeq L/2 ))} \vs 
&&\times2C\,Z_{2}(-\vk_s,\vk_s+\vk_s';a=1) P_{\rm m,\ini}(k_s)d_{\rm m}^{\rm LC,(1)}(\vk_s+\vk_s') \vs\eql{apprrs}
\eea 
$Z_2$ is fixed at $a=1$ when we use a similar approximation as \ec{dt}. In order to recover $d_{\rm m}^{\rm LC,(1)}(\vk_s+\vk_s')$ from the integral, we also take the ratio ${Z_{1}(\vk-\vk_s';a(r))}/{Z_{1}(\vk;a(r))}$ out of the integrand by assuming that $\vk=\vk_s+\vk_s'$ and $r=L/2$ where $L$ is the depth of the survey. By performing these approximations we have proved \ec{2ptddrs}.

Another interesting thing to notice in the redshift space construction is that, the Gaussian noise term $\mathpzc{A}_{\rs}$ is direction-dependent following:
\be 
\mathpzc{A}_{\rs}(\vk_l)\propto(1+f(a=1)\mu_{\vk_l}^{2})
\ee 
One benefit we can take from this numerical result is that the projected detectability in redshift space still will not depend on the direction of $\vk_l$. So the error bars will be the same for both real space and redshift space.

\clearpage 


\bibliography{refs}


\end{document}
