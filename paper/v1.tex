\documentclass[prd,amsmath,amssymb,floatfix,superscriptaddress,nofootinbib,twocolumn]{revtex4-1}

\def\be{\begin{equation}}
\def\ee{\end{equation}}
\def\bea{\begin{eqnarray}}
\def\eea{\end{eqnarray}}
\newcommand{\C}{\rm C}
\newcommand{\LC}{\rm LC}
\newcommand{\ini}{\rm ini}
\newcommand{\vbk}{\vec{\bar{k}}}
\newcommand{\vrr}{\vec{r}}
\newcommand{\vbkp}{\vec{\bar{k'}}}
\newcommand{\bk}{{\bar{k}}}
\newcommand{\vs}{\nonumber\\}
\newcommand{\Var}{\rm Var}
\newcommand{\vk}{\vec{k}}
\newcommand{\ec}[1]{Eq.~(\ref{eq:#1})}
\newcommand{\eec}[2]{Eqs.~(\ref{eq:#1}) and (\ref{eq:#2})}
\newcommand{\Ec}[1]{(\ref{eq:#1})}
\newcommand{\eql}[1]{\label{eq:#1}}
\newcommand{\rf}[1]{\ref{fig:#1}}
\newcommand{\sfig}[2]{
\includegraphics[width=#2]{../plots/#1}
        }
%\newcommand{\sfigg}[2]{
%\includegraphics[width=0.424\paperwidth]{../plots/#1}
%        }
\newcommand{\sfigg}[2]{
\includegraphics[width=#2]{../plots/#1}
        }
\newcommand{\sfiggg}[2]{
\includegraphics[width=0.4\paperwidth]{../plots/#1}
        }
\newcommand{\sfigr}[2]{
\includegraphics[angle=270,origin=c,width=#2]{#1}
        }
\newcommand{\sfigra}[2]{
\includegraphics[angle=90,origin=c,width=#2]{#1}
        }
\newcommand{\Sfig}[2]{
   \begin{figure}[thbp]
   \begin{center}
    \sfig{../plots/#1.pdf}{\columnwidth}
    \caption{{\small #2}}
    \label{fig:#1}
     \end{center}
   \end{figure}
}
\newcommand{\Sfigg}[2]{
   \begin{figure}[thbp]
    \sfigg{../plots/#1.pdf}{.8\paperwidth}
    \caption{{\small #2}}
    \label{fig:#1}
   \end{figure}
}
\newcommand{\Spng}[2]{
   \begin{figure}[thbp]
   \begin{center}
    \sfigg{../plots/#1.png}{\columnwidth}
    \caption{{\small #2}}
    \label{fig:#1}
     \end{center}
   \end{figure}
}

\usepackage[utf8]{inputenc}
\usepackage{graphicx}
\usepackage{amssymb}
\usepackage{simplewick}
\usepackage{amsmath}
\usepackage{bm}
\usepackage{color}
\usepackage{enumitem}
\usepackage[linktocpage=true]{hyperref} 
\hypersetup{
    colorlinks=true,       
    linkcolor=red,         
    citecolor=blue,        
    filecolor=magenta,      
    urlcolor=blue           
}
\usepackage[all]{hypcap} 
\definecolor{darkgreen}{cmyk}{0.85,0.1,1.00,0} 
\definecolor{darkorange}{rgb}{1.0,0.2,0.0}
\newcommand{\scott}[1]{{\color{darkgreen} #1}}
\newcommand{\peikai}[1]{{\color{blue} #1}}
\newcommand{\prvs}[1]{{\color{magenta} #1}}
\newcommand{\AL}[1]{{\color{magenta} AL: #1}}
\newcommand{\MR}[1]{{\color{blue} MR: #1}}
\newcommand{\RC}[1]{{\color{darkorange} #1}}
\newcommand{\nl}{\\ \indent}
\begin{document}
\title{Large Scale Structure Reconstruction with Short-Wavelength Modes: \\Light Cone Formalism and Halo Bias}
\author{\large Peikai Li}
%\affiliation{Department of Physics, Carnegie Mellon University, Pittsburgh, PA 15213, USA}
%\affiliation{McWilliams Center for Cosmology, Carnegie Mellon University, Pittsburgh, PA 15213, USA}
%\author{\large Scott Dodelson}
%\affiliation{Department of Physics, Carnegie Mellon University, Pittsburgh, PA 15213, USA}
%\affiliation{McWilliams Center for Cosmology, Carnegie Mellon University, Pittsburgh, PA 15213, USA}

\date{\today}
\begin{abstract}
\noindent This is the second paper in a series where we propose a method of indirectly measuring large scale structure using information of small scale perturbations. The idea is to use two-point off-diagonal terms of density contrast modes to build a quadratic estimator for long-wavelength modes. We demonstrate in the first paper that our quadratic estimator works well on a dark-matter-only N-body simulation of the snapshot $z=0$. Here we generalize our theory to the case of a light-cone with halo bias taken into consideration. We successfully apply our generalized version of quadratic estimator to a light cone halo catalog of a N-body simulation of size $\sim5.6\,(h^{-1}\,\rm Gpc)^3$. The most distant point in the light cone is at a redshift of $1.4$, which indicates that we might be able to apply our method in the next generation galaxy surveys.
\end{abstract}
\maketitle
\section{Introduction}


\section{Light Cone Formalism}
Here we consider the Fourier transform of the whole light cone, instead of cutting the light cone into thin redshift slices:
\be
\delta^{\LC}_{\rm m}(\vk) :=\int_{V} d^3\vrr  \, \delta_{\rm m}(\vrr) \frac{D_{\ini}}{D_{1}(a(r))}  e^{-i\vk \cdot\vrr} \eql{LC}
\ee
Here $\LC$ stands for "light cone", and $V$ is the whole volome of the light cone. $a(r)$ is the cosmological scale factor. Here we set $a(\vrr=0)$ to be the origin of the light cone thus $a(\vrr)=a(r)$ will not depend of the direction of position $\vrr$. $\delta_{\rm m}(\vrr)$ is the matter density perturbation at location $\vrr$ with redshift $1/(1+a(r))$. $D_1$ is the linear growth factor defined and $D_{\ini} = D_1(a=a_{\ini})$ is the value of $D_1$ at some initial time $a_{\rm ini}$. \peikai{need to define $P^{\rm LC}$ and show $\langle (1)(1) \rangle =0$ somewhere.}
 
We can write the real space matter inhomogeneity field $\delta_{\rm m}(\vrr)$ as:
\be 
\delta_{\rm m}(\vrr)=\delta_{\rm m}(\vrr;a(r))=\int \frac{d^3 \vbk}{(2\pi)^3}e^{i\vbk\cdot\vrr}\delta_{\rm m}(\vbk;a(r))\eql{snp}
\ee 
$\delta_{\rm m}(\vrr)$ is the density contrast at location $\vrr$ of the \textbf{light cone}; while $\delta_{\rm m}(\vrr;a(r))$ is the density contrast at location $\vrr$ of the \textbf{snapshot} at time $a(r)$. \ec{snp} is the inverse Fourier transform of modes $\delta_{\rm m}(\vbk;a(r))$ at a \textbf{fixed} time $a(r)$. If we further transform it back we would get:
\be 
\delta_{\rm m}(\vbk;a(r))=\int \frac{d^3 \vrr}{(2\pi)^3}e^{-i\vbk\cdot\vrr}\delta_{\rm m}(\vrr;\underbrace{a(r)}_{\rm fixed})
\ee 
Here the range of the integration is the whole space of that snapshot. And this equation still holds because we perform the calculation at a fixed time $a(r)$. We have the usual perturbative expansion of modes $\delta_{\rm m}(\vbk;a(r))$ and the time evolution of each order \cite{Bernardeau:2002rev}:
\bea
&&\delta_{\rm m}(\vbk;a(r))=\delta^{(1)}_{\rm m}(\vbk;a(r))+\delta^{(2)}_{\rm m}(\vbk;a(r))+\cdots\vs
&=&\frac{D_1(a(r))}{D_{\rm ini}}\delta^{(1)}_{\rm m,ini}(\vbk)+\bigg[\frac{D_1(a(r))}{D_{\rm ini}}\bigg]^2\delta^{(2)}_{\rm m,ini}(\vbk)+\cdots
\eea 
Similar to our previous work, we still want to compute the off-diagonal term of $\delta^{\LC}_{\rm m}(\vk)$ up to second order. And the first and second order term of $\delta^{\LC}_{\rm m}(\vk)$ will be:
\be 
\delta^{\LC,(\rm i)}_{\rm m}(\vk)=\int_{V} d^3\vrr  \, \int \frac{d^3 \vbk}{(2\pi)^3}e^{i(\vbk-\vk)\cdot\vrr}\delta^{(\rm i)}_{\rm m}(\vbk;a(r))\frac{D_{\ini}}{D_{1}(a(r))}  \eql{exp}
\ee 
with $i=1,2$. Using these expressions, we can compute the two-point correlation of two short-wavelength modes $\delta^{\LC}_{\rm m}(\vk_s)$ and $\delta^{\LC}_{\rm m}(\vk_s')$. We take the squeezed limit $\vk_l=\vk_s+\vk_s'$ with $\vk_s,\vk_s' \gg \vk_l$ and $\vk_l$ is a long-wavelength mode:
\bea 
 &&\langle \delta^{\LC}_{\rm m}(\vk_s)\delta^{\LC}_{\rm m}(\vk_s') \rangle|_{\vk_s+\vk_s'=\vk_l}\vs
 =&&\langle \delta^{\LC,(1)}_{\rm m}(\vk_s)\delta^{\LC,(2)}_{\rm m}(\vk_s') \rangle+\langle \delta^{\LC,(2)}_{\rm m}(\vk_s)\delta^{\LC,(1)}_{\rm m}(\vk_s') \rangle \eql{2p}
\eea 
Substituting \ec{exp} into \ec{2p} and evaluate the first bracket as an example:
\bea 
&&\langle \delta^{\rm LC,(1)}_{\rm m}(\vk_s)\delta^{\rm LC,(2)}_{\rm m}(\vk_s') \rangle \vs 
&=& \int_V d^{3}\vrr\int_{V} d^{3} \vrr' \int \frac{d^{3}\vbk}{(2\pi)^3}\int \frac{d^{3}\vbkp}{(2\pi)^3}\vs
&& e^{i(\vbk\cdot\vrr+\vbkp\cdot\vrr')-i(\vk_s\cdot\vrr+\vk_s'\cdot\vrr')} \frac{D_{\ini}}{D_{1}(a(r))} \frac{D_{\ini}}{D_{1}(a(r'))}  \vs
&\times &\langle \delta_{\rm m}^{(1)}(\vbk;a(r))\delta_{\rm m}^{(2)}(\vbkp;a(r')) \rangle \eql{od}
\eea 
We have computed $\langle \delta_{\rm m,ini}^{(1)}(\vbk)\delta_{\rm m,ini}^{(2)}(\vbkp) \rangle$ in our previous work. We can use the result to further determine the value of the bracket in \ec{od}:
\bea 
&&\langle \delta_{\rm m}^{(1)}(\vbk;a(r))\delta_{\rm m}^{(2)}(\vbkp;a(r')) \rangle \vs
&=&2\frac{D_{1}(a(r))}{D_{\ini}}\bigg[\frac{D_{1}(a(r'))}{D_{\ini}} \bigg]^{2}\langle \delta_{\rm m,ini}^{(1)}(\vbk)\delta_{\rm m,ini}^{(2)}(\vbkp) \rangle  \vs 
&=&2\bigg[\frac{D_{1}(a(r'))}{D_{\ini}} \bigg]^{2}F_{2}(-\vbk,\vbk+\vbkp)P_{\rm m,\ini}(\bk)\delta_{\rm m}^{(1)}(\vbk+\vbkp;a(r)) \eql{odb} \vs
\eea 
where we take advantage of the the definition of the linear growth factor $D_1(a(r))$. $P_{\rm m,ini}$ is the linear matter power spectrum at some initial time $a_{\rm ini}$. Plugging \ec{odb} into \ec{od}, we can see that the only $\vrr'$ dependent integral can be written as (here we simply choose a cube volume):
\be 
\int_{V} d^{3}\vrr'e^{-i(\vbkp-\vk_s')\vrr'}\frac{D_{1}(a(r'))}{D_{\ini}}  \simeq C\,(2\pi)^3\delta_{\rm D}(\vbkp-\vk_s')\eql{dt}
\ee 
since $D_1(a(r'))$ is a slowly varying function. $C$ is a constant and can be further determined via integrating over $\vbk$ on both sides of \ec{dt}. Thus:
\bea 
&& \langle \delta^{\rm LC,(1)}_{\rm m}(\vk_s)\delta^{\rm LC,(2)}_{\rm m}(\vk_s') \rangle \vs 
&=&2C\int_{V} d^{3} \vrr\int \frac{d^{3}\vbk}{(2\pi)^3}F_2(\vbk,-\vbk+\vk_s')P_{\rm m,\ini}(|\vbk-\vk_s'|) \vs
&&\times \frac{D_{\ini}}{D_{1}(a(r))}e^{-i(\vbk-\vk_s-\vk_s')\vrr}\delta_{\rm m}^{(1)}(\vbk;a(r)) \vs
&\simeq & 2C\,F_{2}(-\vk_s,\vk_s+\vk_s')P_{\rm m,\ini}(k_s) \vs
&& \times \int_{V} d^{3} \vrr\int \frac{d^{3}\vbk}{(2\pi)^3}\frac{D_{\ini}}{D_{1}(a(r))}e^{-i(\vbk-\vk_s-\vk_s')\vrr}\delta_{\rm m}^{(1)}(\vbk;a(r))\vs
&= &2C\,F_{2}(-\vk_s,\vk_s+\vk_s')P_{\rm m,\ini}(k_s)\delta_{\rm m}^{\rm LC,(1)}(\vk_s+\vk_s')
\eea 
where in the first step, we perform a redefinition of integration dummy variable. And we use an approximation in the second step in order to get the first order term $\delta_{\rm m}^{\rm LC,(1)}(\vk_l)$ from the integral. 

Again, we have proved that with this construction in \ec{LC}, we can get long-wavelength modes from off-diagonal terms of short-wavelength modes:
\be 
\langle \delta^{\LC}_{\rm m}(\vk_s)\delta^{\LC}_{\rm m}(\vk_s') \rangle|_{\vk_s+\vk_s'=\vk_l} =f(\vec{k}_s,\vec{k}_s'){\delta}^{\rm LC,(1)}_{\rm m}(\vec{k}_l) \eql{2pt}
\ee 
with 
\bea
f(\vec{k}_s,\vec{k}_s')&=&2C\,F_2(-\vec{k}_s,\vec{k}_s+\vec{k}_s')P_{\rm m,ini}(k_s)\vs
&+&2C\,F_2(-\vec{k}_s',\vec{k}_s+\vec{k}_s')P_{\rm m,ini}(k_s')       
\eea 

The quadratic estimator can be formed as:
\be
\hat{\delta}^{\rm LC,(1)}_{\rm m}(\vk_l)=A(\vk_l)\int \frac{d^{3}\vk_s}{(2\pi)^3} g(\vk_s,\vk_s')\delta^{\LC}_{\rm m}(\vk_s)\delta^{\LC}_{\rm m}(\vk_s') \eql{est}
\ee 
with $\vk_s' = \vk_l-\vk_s$ and $g$ being a weighting function. $A$ is the normalization factor determined by requiring that $\langle \hat{\delta}^{\rm LC,(1)}(\vec{k}_l) \rangle={\delta}^{\rm LC,(1)}(\vec{k}_l)$:
\begin{eqnarray}
A(\vec{k}_l)=\bigg[\int \frac{d^3 \vec{k}_s}{(2\pi)^3} g(\vec{k}_s,\vec{k}_s')f(\vec{k}_s,\vec{k}_s')  \bigg]^{-1} \eql{a}
\end{eqnarray}
Similar to our last work, $g$ can be calculated by minimizing the noise term and the result is:
\begin{eqnarray}
&&g(\vec{k}_{s},\vec{k}_{s}')
=\frac{f(\vec{k}_{s},\vec{k}_{s}')}{2P^{\rm LC}_{\rm m}(k_{s})P^{\rm LC}_{\rm m}(k_{s}')}\vs
&=&C\frac{F_2(-\vec{k}_s,\vec{k}_s+\vec{k}_s')P_{\rm m,ini}(k_s)+F_2(-\vec{k}_s',\vec{k}_s+\vec{k}_s')P_{\rm m,ini}(k_s')}{P^{\rm LC}_{\rm m}(k_{s})P^{\rm LC}_{\rm m}(k_{s}')}\vs 
\end{eqnarray} 
 
\section{Halo Bias}
\cite{Desjacques:2018rev} Density contrast of a halo catalog can be written as:
\be 
\delta_{\rm h}(\vrr)=\frac{\sum_{i}M_{i}\delta_{\rm D}(\vrr-\vrr_{i})-\sum_{i}M_{i}/V}{\sum_{i}M_{i}/V}
\ee
Here $i$ is the index for halos. $M_{i}$ and $\vrr_{i}$ is the array for halo mass and position, respectively. We can construct a matter density contrast out of this halo catalog using Tinker bias funtion $b_{1}(M,z)$:
\be 
\delta_{\rm m}(\vrr)=\frac{\sum_{i}\frac{M_{i}}{b_{1}(M_{i},z(r_{i}))}\delta_{\rm D}(\vrr-\vrr_{i})-\sum_{i}\frac{M_{i}}{b_{1}(M_{i},z(r_{i}))}/V}{\sum_{i}M_{i}/V}
\ee 
 such that $\delta_{\rm h}/\delta_{\rm m}\approx b_{1}$.
 
\section{Demonstration with N-body simulation}
We use the MICE Grand Challenge light-cone N-body simulation \cite{Fosalba:2015MI}\cite{Fosalba:2015MII} to demonstrate the power of the estimator.

\acknowledgements
This work has made use of CosmoHub. CosmoHub has been developed by the Port d'Informació Científica (PIC), maintained through a collaboration of the Institut de Física d'Altes Energies (IFAE) and the Centro de Investigaciones Energéticas, Medioambientales y Tecnológicas (CIEMAT), and was partially funded by the "Plan Estatal de Investigación Científica y Técnica y de Innovación" program of the Spanish government.

\bibliography{refs}

\appendix

\section{Non-cube Volume}
For a more generic case, we might have a non-cube volume. Thus:
\be
\delta^{\LC}_{\rm m}(\vk) :=\int_{V} d^3\vrr  \, \delta_{\rm m}(\vrr;a(r))\big[ \frac{D_{\ini}}{D_{1}(a(r))} \big]^{2}f_{V}(\vrr) e^{-i\vk \cdot\vrr}
\ee
 $f_{V}$ is a position-dependent function and its form will be fully determined by the light cone's shape. For a cube volume, simply we have $f_V=1$.

\end{document}
