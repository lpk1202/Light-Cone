\documentclass[prd,amsmath,amssymb,floatfix,superscriptaddress,nofootinbib]{revtex4-1}

\def\be{\begin{equation}}
\def\ee{\end{equation}}
\def\bea{\begin{eqnarray}}
\def\eea{\end{eqnarray}}
\newcommand{\C}{\rm C}
\newcommand{\m}{\rm m}
\newcommand{\rs}{\rm rs}
\newcommand{\LC}{\rm LC}
\newcommand{\ini}{\rm ini}
\newcommand{\vbk}{\vec{\bar{k}}}
\newcommand{\vq}{\vec{q}}
\newcommand{\vrr}{\vec{r}}
\newcommand{\vbkp}{\vec{\bar{k'}}}
\newcommand{\bk}{{\bar{k}}}
\newcommand{\vs}{\nonumber\\}
\newcommand{\Var}{\rm Var}
\newcommand{\vk}{\vec{k}}
\newcommand{\ec}[1]{Eq.~(\ref{eq:#1})}
\newcommand{\eec}[2]{Eqs.~(\ref{eq:#1}) and (\ref{eq:#2})}
\newcommand{\Ec}[1]{(\ref{eq:#1})}
\newcommand{\eql}[1]{\label{eq:#1}}
\newcommand{\rf}[1]{\ref{fig:#1}}
\newcommand{\sfig}[2]{
\includegraphics[width=#2]{../plots/#1}
        }
%\newcommand{\sfigg}[2]{
%\includegraphics[width=0.424\paperwidth]{../plots/#1}
%        }
\newcommand{\sfigg}[2]{
\includegraphics[width=#2]{../plots/#1}
        }
\newcommand{\sfiggg}[2]{
\includegraphics[width=0.4\paperwidth]{../plots/#1}
        }
\newcommand{\sfigr}[2]{
\includegraphics[angle=270,origin=c,width=#2]{#1}
        }
\newcommand{\sfigra}[2]{
\includegraphics[angle=90,origin=c,width=#2]{#1}
        }
\newcommand{\Sfig}[2]{
   \begin{figure}[thbp]
   \begin{center}
    \sfig{../plots/#1.pdf}{\columnwidth}
    \caption{{\small #2}}
    \label{fig:#1}
     \end{center}
   \end{figure}
}
\newcommand{\Sfigg}[2]{
   \begin{figure}[thbp]
    \sfigg{../plots/#1.pdf}{.8\paperwidth}
    \caption{{\small #2}}
    \label{fig:#1}
   \end{figure}
}
\newcommand{\Spng}[2]{
   \begin{figure}[thbp]
   \begin{center}
    \sfigg{../plots/#1.png}{\columnwidth}
    \caption{{\small #2}}
    \label{fig:#1}
     \end{center}
   \end{figure}
}
\usepackage{calligra}
\DeclareMathAlphabet{\mathcalligra}{T1}{calligra}{m}{n}
\DeclareMathAlphabet{\mathpzc}{OT1}{pzc}{m}{it}
\usepackage[utf8]{inputenc}
\usepackage{graphicx}
\usepackage{amssymb}
\usepackage{simplewick}
\usepackage{amsmath}
\usepackage{bm}
\usepackage{color}
\usepackage{enumitem}
\usepackage[linktocpage=true]{hyperref} 
\hypersetup{
    colorlinks=true,       
    linkcolor=red,         
    citecolor=blue,        
    filecolor=magenta,      
    urlcolor=blue           
}
\usepackage[all]{hypcap} 
\definecolor{darkgreen}{cmyk}{0.85,0.1,1.00,0} 
\definecolor{darkorange}{rgb}{1.0,0.2,0.0}
\newcommand{\scott}[1]{{\color{darkgreen} \, [Scott: #1]}}
\newcommand{\peikai}[1]{{\color{blue} #1}}
\newcommand{\prvs}[1]{{\color{magenta} #1}}
\newcommand{\AL}[1]{{\color{magenta} AL: #1}}
\newcommand{\MR}[1]{{\color{blue} MR: #1}}
\newcommand{\RC}[1]{{\color{darkorange} #1}}
\newcommand{\nl}{\\ \indent}
\newcommand\dmh{\delta_{\rm m}^{\rm h}}
\newcommand\hdmh{\hat{\delta}_{\rm m}^{\rm h}}
\begin{document}
\title{Angular Clustering}
%\author{\large Peikai Li}
%\affiliation{Department of Physics, Carnegie Mellon University, Pittsburgh, PA 15213, USA}
%\affiliation{McWilliams Center for Cosmology, Carnegie Mellon University, Pittsburgh, PA 15213, USA}
%\author{\large Rupert A. C. Croft}
%\affiliation{Department of Physics, Carnegie Mellon University, Pittsburgh, PA 15213, USA}
%\affiliation{McWilliams Center for Cosmology, Carnegie Mellon University, Pittsburgh, PA 15213, USA}
%\author{\large Scott Dodelson}
%\affiliation{Department of Physics, Carnegie Mellon University, Pittsburgh, PA 15213, USA}
%\affiliation{McWilliams Center for Cosmology, Carnegie Mellon University, Pittsburgh, PA 15213, USA}
%
%\date{\today}
%\begin{abstract}
%\noindent This is the second paper in a series where we propose a method of indirectly measuring large scale structure using information from small scale perturbations. The idea is to build a quadratic estimator from small scale modes that provides a map of structure on large scales.
%We demonstrated in the first paper that the quadratic estimator works well on a dark-matter-only N-body simulation at a snapshot of $z=0$. Here we generalize the theory to the case of a light-cone with halo bias and redshift space distortions taken into consideration. We successfully apply the generalized version of the quadratic estimator to a light cone halo catalog of an N-body simulation of size $\sim5.6\,(h^{-1}\,\rm Gpc)^3$. The most distant point in the light cone is at a redshift of $1.4$, which indicates that we might be able to apply our method to next generation galaxy surveys.
%%\noindent This is the second paper in a series where we propose a method of indirectly measuring large scale structure using information of small scale perturbations. The idea is to use two-point off-diagonal terms of density contrast modes to build a quadratic estimator for long-wavelength modes. We demonstrate in the first paper that our quadratic estimator works well on a dark-matter-only N-body simulation of the snapshot $z=0$. Here we generalize our theory to the case of a light-cone with halo bias taken into consideration. We successfully apply our generalized version of the quadratic estimator to a light cone halo catalog of a N-body simulation of size $\sim5.6\,(h^{-1}\,\rm Gpc)^3$. The most distant point in the light cone is at a redshift of $1.4$, which indicates that we might be able to apply our method to next generation galaxy surveys.
%%
%%\noindent This is the second paper in a series where we propose a method of indirectly measuring large scale structure using information of small scale perturbations. The idea is to use two-point off-diagonal terms of density contrast modes to build a quadratic estimator for long-wavelength modes. We demonstrate in the first paper that our quadratic estimator works well on a dark-matter-only N-body simulation of the snapshot $z=0$. Here we generalize our theory to the case of a light cone with halo bias taken into consideration. We successfully apply our generalized version of the quadratic estimator to a light cone halo catalog of a N-body simulation of size $\sim5.6\,(h^{-1}\,\rm Gpc)^3$. The most distant point in the light cone is at a redshift of $1.4$, which indicates that we might be able to apply our method to next generation galaxy surveys.
%\end{abstract}
\maketitle
\section{Estimator}

\subsection{Take 1}
Consider measuring a 2-point function on angular scales $\theta\ll R$ in a region of anuglar size $R^2$. Set the mask $M_R(\vec\theta)$ to be zero outside of the large $R$ region and one inside, and similarly $\delta_R(\vec\theta) = \delta(\theta)M_R(\vec\theta)$ vanishes outside the mask. Then, in the relevant region, the 2-point function is
\be
w_R(\theta)= \int d^2\theta_1 \int d^2\theta_2 \delta_R(\vec\theta_1) \delta_R(\vec\theta_2) \Theta_\theta(|\vec\theta_1-\vec\theta_2|)
\ee
where $\Theta_\theta()$ is non-zero only when its argument falls in the $\theta$ bin.
The product in real space becomes a convoultion in Fourier space:
\be
\delta_R(\vec l) = \int \frac{d^2l'}{(2\pi)^2}\, \delta(\vec l') M_R(\vec l-\vec l') 
.\ee
Therefore,
\bea
w_R(\theta) &=& \int d^2\theta_1 \int d^2\theta_2 \int \frac{d^2l_1}{(2\pi)^2}\, e^{i\vec l_1\cdot \vec\theta_1}\int \frac{d^2l'}{(2\pi)^2}\, \delta(\vec l') M_R(\vec l_1-\vec l') 
\vs
&\times&
\int \frac{d^2l_2}{(2\pi)^2}\, e^{i\vec l_2\cdot \vec\theta_2}\int \frac{d^2l''}{(2\pi)^2}\, \delta(\vec l'') M_R(\vec l_2-\vec l'') 
 \Theta_\theta(|\vec\theta_1-\vec\theta_2|).
\eea
Define $\vec\theta_\pm \equiv (\vec\theta_1\pm\vec\theta_2)$, so that $\vec\theta_1=\frac12(\vec\theta_++\vec\theta_-)$ and $\vec\theta_2=\frac12(\vec\theta_+-\vec\theta_-)$. The Jacobian is $4$ so,
\be
 \int d^2\theta_1 \int d^2\theta_2  =  \frac14\int d^2\theta_+ \int d^2\theta_- 
\ee
and
\bea
w_R(\theta) &=& \frac14\int d^2\theta_+ \int d^2\theta_- \int \frac{d^2l_1}{(2\pi)^2}\, e^{\frac{i}2\vec l_1\cdot (\vec\theta_++\vec\theta_-)}\int \frac{d^2l'}{(2\pi)^2}\, \delta(\vec l') M_R(\vec l_1-\vec l') 
\vs
&\times&
\int \frac{d^2l_2}{(2\pi)^2}\, e^{\frac{i}2\vec l_2\cdot (\vec\theta_+-\vec\theta_-)}\int \frac{d^2l''}{(2\pi)^2}\, \delta(\vec l'') M_R(\vec l_2-\vec l'') 
 \Theta_\theta(\theta_-).
\eea
The $\vec\theta_+$ integral then gives a delta function in $\vec l_1+\vec l_2$, so
\be
w_R(\theta) =  \int d^2\theta_- \int \frac{d^2l_1}{(2\pi)^2}\, e^{i\vec l_1\cdot\vec\theta_-}\, \int \frac{d^2l'}{(2\pi)^2}\, \delta(\vec l') M_R(\vec l_1-\vec l') 
\,\int \frac{d^2l''}{(2\pi)^2}\, \delta(\vec l'') M_R(\vec l_1+\vec l'') 
 \Theta_\theta(\theta_-).
\ee
The $\theta_-$ integral then just gives the complex conjugate of the Fourier transform of $\Theta_\theta$ so
\be
w_R(\theta) =  \int \frac{d^2l_1}{(2\pi)^2}\, \tilde \Theta_\theta^*(\vec l_1)\, \int \frac{d^2l'}{(2\pi)^2}\, M_R(\vec l_1-\vec l') 
\,\int \frac{d^2l''}{(2\pi)^2}\, M_R(\vec l_1+\vec l'') \delta(\vec l') \delta(\vec l'') 
.
\ee

Fig.~\rf{mask} shows the mask in Fourier space for a one degree patch. Large $l$ modes are clearly suppressed. Analytically,
\be
M_R(l) = \frac{2\pi J_1(lR)}{l}.\ee
Also,
\be
\int \frac{d^2l}{(2\pi)^2} \, M_R^2(l) =A\ee
the area observed.
\Spng{mask}{The mask in Fourier space for a one degree patch.}

The zeroth order term is obtained by setting
\be
\langle\delta(\vec l)\delta(\vec l')\rangle = (2\pi)^2\delta^2(\vec l+\vec l') C_l\ee
so that 
\be
w_R(\theta) =  \int \frac{d^2l_1}{(2\pi)^2}\, \tilde \Theta_\theta^*(\vec l_1)\, \int \frac{d^2l'}{(2\pi)^2}\, M_R^2(\vec l_1-\vec l') 
\, C_{l'}
.
\ee

\subsection{Take 2}

Suppose that as in the 3D case, it was true that 
\be
\langle \delta(\vec L-\vec l)\delta(\vec l)\rangle  = R(\vec L,\vec l) \delta (\vec L)\ee
where the response function $R$ is calculable. Here $\delta$ is taken to mean any 2D field that can be measure, e.g., shear or $\kappa$ or galaxy overdensity. 
How would this translate to real space? On the face of it, simply take the Fourier transform of both sides:
\be
\hat \delta(\vec\theta) = \int \frac{d^2L}{(2\pi)^2}\, e^{i\vec L\cdot\vec\theta} \, \frac{\delta(\vec L-\vec l)\delta(\vec l)}{R(\vec L,\vec l) }
.\ee
There are two tweaks to make this more realistic. First, this expression is in principle true for any large value of $l$ and presumably the $l$ dependence drops out when the expectation value on the right is taken. Ideally, though, one sums over many values of $\vec l$ with some weighting factor, which presumably depends on both $\vec l$ and $\vec L$. We can absorb $R$ into this weighting factor, call it $F$, and form an estimator
\be
\hat \delta(\vec\theta) = \int \frac{d^2L}{(2\pi)^2}\, e^{i\vec L\cdot\vec\theta} \, \int \frac{d^2l}{(2\pi)^2}\, F(\vec L,\vec l)\, \delta(\vec L-\vec l)\delta(\vec l)
.\ee
The second tweak is to enforce the idea that we are trying to create an estimator of $\delta(\vec\theta)$ on large scales. We can make this more explicit by integrating both sides of $\vec\theta$ centered at some value $\vec\Theta$ with a smoothing function $W_\phi(\vec\Theta,\vec\theta)$. For example, $W_\phi$ might be a tophat that is equal to one whenever $|\vec\Theta-\vec\theta|<\phi$ and zero otherwise. In this case,
\be
\hat \delta(\vec\Theta) = \int d^2\theta W_\phi(\vec\Theta,\vec\theta)\, \int \frac{d^2L}{(2\pi)^2}\, e^{i\vec L\cdot\vec\theta} \, \int \frac{d^2l}{(2\pi)^2}\, F(\vec L,\vec l)\, \delta(\vec L-\vec l)\delta(\vec l)
.\ee
With $\vec\theta=\vec\Theta+\vec\alpha$, the angular integral then becomes
\be
e^{i\vec L\cdot\vec\Theta}\, \int d^2\alpha e^{i\vec L\cdot\vec\alpha} W_\phi(\vec\alpha)
= e^{i\vec L\cdot\vec\Theta}\, \int_0^\phi d\alpha \alpha \int_0^{2\pi} d\beta e^{iL\alpha\cos\beta}.\ee
The inner integral is $2\pi J_0(L\alpha)$. When doing the full integral, the result should be a low pass filter, cutting off all $L$-modes with $L>1/\phi$, the smoothing scale.


To see what this might look like in Fourier space, expand the $\delta$'s in the integrand as inverse Fourier transforms. Then,
\bea
\hat \delta(\vec\theta) &=& \int \frac{d^2L}{(2\pi)^2}\, e^{i\vec L\cdot\vec\theta} \, \int \frac{d^2l}{(2\pi)^2}\, W(\vec L, \vec l)
\vs
&\times&\int d^2\theta_1 e^{-i\vec\theta_1\cdot(\vec L - \vec l)} \,
\int d^2\theta_2 e^{-i\vec\theta_2\cdot\vec l} \delta(\vec\theta_1)\delta(\vec\theta_2)
\eea



It is possible then to sum over many small scale $l$ modes in order to create an optimal estimator of $\delta(\vec L)$. Let's write that generally as
\be
\hat\delta(\vec L) = \int \frac{d^2l}{(2\pi)^2}\, W(\vec L, \vec l) \delta(\vec L-\vec l)\delta(\vec l)
\ee
where $W$ is normalized so that the estimator is unbiased.


We can take the individual Fourier transforms of each of these
\be
\frac1{R(\vec L,\vec l)}\, \int d^2\theta \int d^2\theta' e^{-i(\vec L-\vec l)\cdot\vec\theta}\, e^{-i\vec l\cdot\vec\theta'} \, \langle \delta(\vec\theta)\delta(\vec\theta') \rangle
= \int d^2\Theta\, e^{-i\vec L\cdot\vec\Theta} \delta(\vec \Theta)
\ee


\section{Projection} \label{sec1}
Consider an angular measurement
\be
f(\vec\theta) = \int d\chi W_f(\chi) \delta(\chi\vec\theta,\chi)
\ee
where I work in the Limber approximation. Then the Fourier transform of this field is
\bea
\tilde f(\vec l) &=& \int d\chi W_f(\chi) \int \frac{d^2l}{(2\pi)^2}\, e^{-i\vec l\cdot\vec\theta}\,\int \frac{d^3k}{(2\pi)^3}\, e^{i\chi\vec k_\perp\cdot \vec \theta} e^{ik_z\chi} \tilde\delta(\vec k;\chi)
\vs
&=& 
\int d\chi \frac{W_f(\chi) }{\chi^2} \, \int \frac{dk_z}{2\pi}\, e^{ik_z\chi}  \tilde\delta(\vec l/\chi;k_z;\chi).
\eea
Then the two-point function becomes:
\be
\langle \tilde f(\vec l)  \tilde g(\vec l') \rangle 
= \int d\chi \frac{W_f(\chi) }{\chi^2} \, \int d\chi \frac{W_g(\chi) }{\chi^2} \, \int \frac{dk_z}{2\pi}\, \int \frac{dk_z'}{2\pi}\, e^{ik_z\chi+ik_z'\chi'}  
\langle \tilde\delta(\vec l/\chi;k_z;\chi) \tilde\delta(\vec l'/\chi';k_z';\chi')\rangle\eql{twop}
\ee
We now write
\newcommand\de[1]{\tilde\delta^{(#1)}}
\be
\delta = \de1+\de2
\ee
where
\be
\de1 = \de{L} D(\chi)
\ee
the linear overdensity at some early time propagated forward with the growth factor. Then,
\be
\de2(\vec k;\chi) = \int \frac{d^3k_1}{(2\pi)^3}\, F_2(\vec k_1,\vec k-\vec k_1) \de1(\vec k_1;\chi) \de1(\vec k-\vec k_1;\chi).
\ee
We know that the linear terms in \ec{twop} will lead to the diagonal power spectrum. We are interested in the off-diagonal terms, when $\vec l\ne-\vec l'$. For that, we need to explore the contribution from the $\langle \de1\de2\rangle$ terms. Consider one of these:
\bea
\langle \tilde f(\vec l)  \tilde g(\vec l') \rangle_{l\ne l'} &=& \int d\chi \frac{W_f(\chi) }{\chi^2} \, \int d\chi \frac{W_g(\chi) }{\chi^2} \, \int \frac{dk_z}{2\pi}\, \int \frac{dk_z'}{2\pi}\, e^{ik_z\chi+ik_z'\chi'}  
\langle\de2(\vec l/\chi;k_z;\chi) \de1(\vec l'/\chi';k_z';\chi')\rangle \vs
&=&
\int d\chi \frac{W_f(\chi) }{\chi^2} \, \int d\chi \frac{W_g(\chi) }{\chi^2} \, \int \frac{dk_z}{2\pi}\, \int \frac{dk_z'}{2\pi}\, e^{ik_z\chi+ik_z'\chi'}  
\vs
&&\times 
\int \frac{d^3k_1}{(2\pi)^3}\, F_2(\vec k_1,\vec k-\vec k_1) \langle \de1(\vec k_1;\chi) \de1(\vec k-\vec k_1;\chi) \de1(\vec l'/\chi';k_z';\chi')\rangle 
\eea
where on the last line $\vec k = [\vec l/\chi,k_z]$. It is the last two $\delta$'s that are small scale perturbations and we want to understand their behavior in the presence of a long-wavelength perturbation $\de1(\vec k_1)$. When contracting those last two, we get a delta function that constrains:
\bea
\vec l/\chi - \vec k_{1,\perp} + \vec l'/\chi' &=& 0
\vs
k_z-k_{1,z} + k_z' &=& 0
\eea
or
\be
\vec k_1 = [\vec l/\chi+\vec l'/\chi',k_z+k_z'].
\ee
So, we are left with
\be
\langle \tilde f(\vec l)  \tilde g(\vec l') \rangle_{l\ne l'} =
\int d\chi \frac{W_f(\chi) D(\chi)}{\chi^2} \, \int d\chi'\frac{W_g(\chi') D(\chi')}{\chi'^2} \, \int \frac{dk_z}{2\pi}\,  \int \frac{dk_z'}{2\pi}\, e^{ik_z\chi+ik_z'\chi'}  \de1(\vec k_1;\chi)
F_2(\vec k_1,\vec k-\vec k_1) P(l'/\chi',k_z') \ee

One way to make progress here would be to cross-correlate this with the true 3D density field, $\de{L}(\vec K)$. This is not as crazy as it sounds since galaxies in a spectroscopic survey presumably trace the overdensity with a simple bias factor on large scales and the density is linear on those scales. In that case, the 3-point function would lead to a 3D Dirac delta function, eliminating 3 of the integrals. The delta function would be
\be
\delta^3(\vec K + \vec k_1) = \delta^2(\vec l/\chi + \vec l'/\chi'+\vec K_\perp) \delta(k_z+k_z'+K_z).
\ee
Going slowly, the $k_z'$ integral goes away leaving
\bea
\langle \de{L}(\vec K;\bar\chi)) \tilde f(\vec l)  \tilde g(\vec l') \rangle &=& P(K) D(\bar\chi)
\int d\chi \frac{W_f(\chi) D^2(\chi)}{\chi^2} \, \int d\chi'\frac{W_g(\chi') D(\chi')}{\chi'^2} \, (2\pi)^2  \delta^2(\vec l/\chi + \vec l'/\chi'+\vec K_\perp) 
\vs
&&\times
\int \frac{dk_z}{2\pi}\,  e^{ik_z\chi-i(k_z+K_z)\chi'}  
F_2(-\vec K,\vec k+\vec K) P(l'/\chi',-k_z-K_z) 
\eea
where again 
\be
\vec k = [\vec l/\chi,k_z]
\ee
and $\bar\chi$ is simply the mean distance of galaxies for which you think you have measured $\delta(\vec K)$.
Consider the $\chi'$ integral; use one of the remaining delta functions to do it:
\be
\delta(l_x/\chi+l'_x/\chi'+K_x) =\chi'^2 \frac{\delta(\chi'+ l'_x/[l_x/\chi+ K_x])}{l'_x}
\ee
\bibliography{refs}


\end{document}
