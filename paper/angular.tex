\documentclass[prd,amsmath,amssymb,floatfix,superscriptaddress,nofootinbib]{revtex4-1}

\def\be{\begin{equation}}
\def\ee{\end{equation}}
\def\bea{\begin{eqnarray}}
\def\eea{\end{eqnarray}}
\newcommand{\C}{\rm C}
\newcommand{\m}{\rm m}
\newcommand{\rs}{\rm rs}
\newcommand{\LC}{\rm LC}
\newcommand{\ini}{\rm ini}
\newcommand{\vbk}{\vec{\bar{k}}}
\newcommand{\vq}{\vec{q}}
\newcommand{\vrr}{\vec{r}}
\newcommand{\vbkp}{\vec{\bar{k'}}}
\newcommand{\bk}{{\bar{k}}}
\newcommand{\vs}{\nonumber\\}
\newcommand{\Var}{\rm Var}
\newcommand{\vk}{\vec{k}}
\newcommand{\ec}[1]{Eq.~(\ref{eq:#1})}
\newcommand{\eec}[2]{Eqs.~(\ref{eq:#1}) and (\ref{eq:#2})}
\newcommand{\Ec}[1]{(\ref{eq:#1})}
\newcommand{\eql}[1]{\label{eq:#1}}
\newcommand{\rf}[1]{\ref{fig:#1}}
\newcommand{\sfig}[2]{
\includegraphics[width=#2]{../plots/#1}
        }
%\newcommand{\sfigg}[2]{
%\includegraphics[width=0.424\paperwidth]{../plots/#1}
%        }
\newcommand{\sfigg}[2]{
\includegraphics[width=#2]{../plots/#1}
        }
\newcommand{\sfiggg}[2]{
\includegraphics[width=0.4\paperwidth]{../plots/#1}
        }
\newcommand{\sfigr}[2]{
\includegraphics[angle=270,origin=c,width=#2]{#1}
        }
\newcommand{\sfigra}[2]{
\includegraphics[angle=90,origin=c,width=#2]{#1}
        }
\newcommand{\Sfig}[2]{
   \begin{figure}[thbp]
   \begin{center}
    \sfig{../plots/#1.pdf}{\columnwidth}
    \caption{{\small #2}}
    \label{fig:#1}
     \end{center}
   \end{figure}
}
\newcommand{\Sfigg}[2]{
   \begin{figure}[thbp]
    \sfigg{../plots/#1.pdf}{.8\paperwidth}
    \caption{{\small #2}}
    \label{fig:#1}
   \end{figure}
}
\newcommand{\Spng}[2]{
   \begin{figure}[thbp]
   \begin{center}
    \sfigg{../plots/#1.png}{\columnwidth}
    \caption{{\small #2}}
    \label{fig:#1}
     \end{center}
   \end{figure}
}
\usepackage{calligra}
\DeclareMathAlphabet{\mathcalligra}{T1}{calligra}{m}{n}
\DeclareMathAlphabet{\mathpzc}{OT1}{pzc}{m}{it}
\usepackage[utf8]{inputenc}
\usepackage{graphicx}
\usepackage{amssymb}
\usepackage{simplewick}
\usepackage{amsmath}
\usepackage{bm}
\usepackage{color}
\usepackage{enumitem}
\usepackage[linktocpage=true]{hyperref} 
\hypersetup{
    colorlinks=true,       
    linkcolor=red,         
    citecolor=blue,        
    filecolor=magenta,      
    urlcolor=blue           
}
\usepackage[all]{hypcap} 
\definecolor{darkgreen}{cmyk}{0.85,0.1,1.00,0} 
\definecolor{darkorange}{rgb}{1.0,0.2,0.0}
\newcommand{\scott}[1]{{\color{darkgreen} \, [Scott: #1]}}
\newcommand{\peikai}[1]{{\color{blue} #1}}
\newcommand{\prvs}[1]{{\color{magenta} #1}}
\newcommand{\AL}[1]{{\color{magenta} AL: #1}}
\newcommand{\MR}[1]{{\color{blue} MR: #1}}
\newcommand{\RC}[1]{{\color{darkorange} #1}}
\newcommand{\nl}{\\ \indent}
\newcommand\dmh{\delta_{\rm m}^{\rm h}}
\newcommand\hdmh{\hat{\delta}_{\rm m}^{\rm h}}
\begin{document}
\title{Angular Clustering}
%\author{\large Peikai Li}
%\affiliation{Department of Physics, Carnegie Mellon University, Pittsburgh, PA 15213, USA}
%\affiliation{McWilliams Center for Cosmology, Carnegie Mellon University, Pittsburgh, PA 15213, USA}
%\author{\large Rupert A. C. Croft}
%\affiliation{Department of Physics, Carnegie Mellon University, Pittsburgh, PA 15213, USA}
%\affiliation{McWilliams Center for Cosmology, Carnegie Mellon University, Pittsburgh, PA 15213, USA}
%\author{\large Scott Dodelson}
%\affiliation{Department of Physics, Carnegie Mellon University, Pittsburgh, PA 15213, USA}
%\affiliation{McWilliams Center for Cosmology, Carnegie Mellon University, Pittsburgh, PA 15213, USA}
%
%\date{\today}
%\begin{abstract}
%\noindent This is the second paper in a series where we propose a method of indirectly measuring large scale structure using information from small scale perturbations. The idea is to build a quadratic estimator from small scale modes that provides a map of structure on large scales.
%We demonstrated in the first paper that the quadratic estimator works well on a dark-matter-only N-body simulation at a snapshot of $z=0$. Here we generalize the theory to the case of a light-cone with halo bias and redshift space distortions taken into consideration. We successfully apply the generalized version of the quadratic estimator to a light cone halo catalog of an N-body simulation of size $\sim5.6\,(h^{-1}\,\rm Gpc)^3$. The most distant point in the light cone is at a redshift of $1.4$, which indicates that we might be able to apply our method to next generation galaxy surveys.
%%\noindent This is the second paper in a series where we propose a method of indirectly measuring large scale structure using information of small scale perturbations. The idea is to use two-point off-diagonal terms of density contrast modes to build a quadratic estimator for long-wavelength modes. We demonstrate in the first paper that our quadratic estimator works well on a dark-matter-only N-body simulation of the snapshot $z=0$. Here we generalize our theory to the case of a light-cone with halo bias taken into consideration. We successfully apply our generalized version of the quadratic estimator to a light cone halo catalog of a N-body simulation of size $\sim5.6\,(h^{-1}\,\rm Gpc)^3$. The most distant point in the light cone is at a redshift of $1.4$, which indicates that we might be able to apply our method to next generation galaxy surveys.
%%
%%\noindent This is the second paper in a series where we propose a method of indirectly measuring large scale structure using information of small scale perturbations. The idea is to use two-point off-diagonal terms of density contrast modes to build a quadratic estimator for long-wavelength modes. We demonstrate in the first paper that our quadratic estimator works well on a dark-matter-only N-body simulation of the snapshot $z=0$. Here we generalize our theory to the case of a light cone with halo bias taken into consideration. We successfully apply our generalized version of the quadratic estimator to a light cone halo catalog of a N-body simulation of size $\sim5.6\,(h^{-1}\,\rm Gpc)^3$. The most distant point in the light cone is at a redshift of $1.4$, which indicates that we might be able to apply our method to next generation galaxy surveys.
%\end{abstract}
\maketitle

\section{Introduction} \label{sec1}
Consider an angular measurement
\be
f(\vec\theta) = \int d\chi W_f(\chi) \delta(\chi\vec\theta,\chi)
\ee
where I work in the Limber approximation. Then the Fourier transform of this field is
\bea
\tilde f(\vec l) &=& \int d\chi W_f(\chi) \int \frac{d^2l}{(2\pi)^2}\, e^{-i\vec l\cdot\vec\theta}\,\int \frac{d^3k}{(2\pi)^3}\, e^{i\chi\vec k_\perp\cdot \vec \theta} e^{ik_z\chi} \tilde\delta(\vec k;\chi)
\vs
&=& 
\int d\chi \frac{W_f(\chi) }{\chi^2} \, \int \frac{dk_z}{2\pi}\, e^{ik_z\chi}  \tilde\delta(\vec l/\chi;k_z;\chi).
\eea
Then the two-point function becomes:
\be
\langle \tilde f(\vec l)  \tilde g(\vec l') \rangle 
= \int d\chi \frac{W_f(\chi) }{\chi^2} \, \int d\chi \frac{W_g(\chi) }{\chi^2} \, \int \frac{dk_z}{2\pi}\, \int \frac{dk_z'}{2\pi}\, e^{ik_z\chi+ik_z'\chi'}  
\langle \tilde\delta(\vec l/\chi;k_z;\chi) \tilde\delta(\vec l'/\chi';k_z';\chi')\rangle\eql{twop}
\ee
We now write
\newcommand\de[1]{\tilde\delta^{(#1)}}
\be
\delta = \de1+\de2
\ee
where
\be
\de1 = \de{L} D(\chi)
\ee
the linear overdensity at some early time propagated forward with the growth factor. Then,
\be
\de2(\vec k;\chi) = \int \frac{d^3k_1}{(2\pi)^3}\, F_2(\vec k_1,\vec k-\vec k_1) \de1(\vec k_1;\chi) \de1(\vec k-\vec k_1;\chi).
\ee
We know that the linear terms in \ec{twop} will lead to the diagonal power spectrum. We are interested in the off-diagonal terms, when $\vec l\ne-\vec l'$. For that, we need to explore the contribution from the $\langle \de1\de2\rangle$ terms. Consider one of these:
\bea
\langle \tilde f(\vec l)  \tilde g(\vec l') \rangle_{l\ne l'} &=& \int d\chi \frac{W_f(\chi) }{\chi^2} \, \int d\chi \frac{W_g(\chi) }{\chi^2} \, \int \frac{dk_z}{2\pi}\, \int \frac{dk_z'}{2\pi}\, e^{ik_z\chi+ik_z'\chi'}  
\langle\de2(\vec l/\chi;k_z;\chi) \de1(\vec l'/\chi';k_z';\chi')\rangle \vs
&=&
\int d\chi \frac{W_f(\chi) }{\chi^2} \, \int d\chi \frac{W_g(\chi) }{\chi^2} \, \int \frac{dk_z}{2\pi}\, \int \frac{dk_z'}{2\pi}\, e^{ik_z\chi+ik_z'\chi'}  
\vs
&&\times 
\int \frac{d^3k_1}{(2\pi)^3}\, F_2(\vec k_1,\vec k-\vec k_1) \langle \de1(\vec k_1;\chi) \de1(\vec k-\vec k_1;\chi) \de1(\vec l'/\chi';k_z';\chi')\rangle 
\eea
where on the last line $\vec k = [\vec l/\chi,k_z]$. It is the last two $\delta$'s that are small scale perturbations and we want to understand their behavior in the presence of a long-wavelength perturbation $\de1(\vec k_1)$. When contracting those last two, we get a delta function that constrains:
\bea
\vec l/\chi - \vec k_{1,\perp} + \vec l'/\chi' &=& 0
\vs
k_z-k_{1,z} + k_z' &=& 0
\eea
or
\be
\vec k_1 = [\vec l/\chi+\vec l'/\chi',k_z+k_z'].
\ee
So, we are left with
\be
\langle \tilde f(\vec l)  \tilde g(\vec l') \rangle_{l\ne l'} =
\int d\chi \frac{W_f(\chi) D(\chi)}{\chi^2} \, \int d\chi'\frac{W_g(\chi') D(\chi')}{\chi'^2} \, \int \frac{dk_z}{2\pi}\,  \int \frac{dk_z'}{2\pi}\, e^{ik_z\chi+ik_z'\chi'}  \de1(\vec k_1;\chi)
F_2(\vec k_1,\vec k-\vec k_1) P(l'/\chi',k_z') \ee

One way to make progress here would be to cross-correlate this with the true 3D density field, $\de{L}(\vec K)$. This is not as crazy as it sounds since galaxies in a spectroscopic survey presumably trace the overdensity with a simple bias factor on large scales and the density is linear on those scales. In that case, the 3-point function would lead to a 3D Dirac delta function, eliminating 3 of the integrals. The delta function would be
\be
\delta^3(\vec K + \vec k_1) = \delta^2(\vec l/\chi + \vec l'/\chi'+\vec K_\perp) \delta(k_z+k_z'+K_z).
\ee
Going slowly, the $k_z'$ integral goes away leaving
\bea
\langle \de{L}(\vec K;\bar\chi)) \tilde f(\vec l)  \tilde g(\vec l') \rangle &=& P(K) D(\bar\chi)
\int d\chi \frac{W_f(\chi) D^2(\chi)}{\chi^2} \, \int d\chi'\frac{W_g(\chi') D(\chi')}{\chi'^2} \, (2\pi)^2  \delta^2(\vec l/\chi + \vec l'/\chi'+\vec K_\perp) 
\vs
&&\times
\int \frac{dk_z}{2\pi}\,  e^{ik_z\chi-i(k_z+K_z)\chi'}  
F_2(-\vec K,\vec k+\vec K) P(l'/\chi',-k_z-K_z) 
\eea
where again 
\be
\vec k = [\vec l/\chi,k_z]
\ee
and $\bar\chi$ is simply the mean distance of galaxies for which you think you have measured $\delta(\vec K)$.
Consider the $\chi'$ integral; use one of the remaining delta functions to do it:
\be
\delta(l_x/\chi+l'_x/\chi'+K_x) =\chi'^2 \frac{\delta(\chi'+ l'_x/[l_x/\chi+ K_x])}{l'_x}
\ee
\bibliography{refs}


\end{document}
