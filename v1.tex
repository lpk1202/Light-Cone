\documentclass{article}
\usepackage[utf8]{inputenc}
\usepackage{graphicx}
\usepackage{amsmath}
\usepackage{scalerel,stackengine}
\stackMath
\newcommand\reallywidehat[1]{%
\savestack{\tmpbox}{\stretchto{%
  \scaleto{%
    \scalerel*[\widthof{\ensuremath{#1}}]{\kern-.6pt\bigwedge\kern-.6pt}%
    {\rule[-\textheight/2]{1ex}{\textheight}}%WIDTH-LIMITED BIG WEDGE
  }{\textheight}% 
}{0.5ex}}%
\stackon[1pt]{#1}{\tmpbox}%
}
\parskip 1ex

\def\be{\begin{equation}}

\def\ee{\end{equation}}

\def\bea{\begin{eqnarray}}

\def\eea{\end{eqnarray}}

\newcommand{\vs}{\nonumber\\}
\newcommand{\LC}{\rm LC}
\newcommand{\vk}{\vec{k}}
\newcommand{\vr}{\vec{r}}
\newcommand{\ini}{\rm ini}
\newcommand{\vbk}{\vec{\bar{k}}}
\newcommand{\vbkp}{\vec{\bar{k'}}}
\newcommand{\bk}{{\bar{k}}}
\newcommand{\ec}[1]{Eq.~(\ref{eq:#1})}
\newcommand{\eec}[2]{Eqs.~(\ref{eq:#1}) and (\ref{eq:#2})}
\newcommand{\Ec}[1]{(\ref{eq:#1})}
\newcommand{\eql}[1]{\label{eq:#1}}
\title{Large Scale Structure Reconstruction with Short-Wavelength Modes: Light Cone Formalism}
\author{Peikai Li}

\begin{document}

\maketitle

\section{Light Cone Formalism}
Definition of Fourier modes of a whole light cone:
\be
\delta^{\LC}(\vk) = \int_{V} d^3 \vr\, \delta(\vr;a(r))\big[ \frac{D_{\ini}}{D_{1}(a(r))} \big]^{2}f_{V}(\vr) e^{-i\vk \cdot\vr} \eql{LC}
\ee
Here $\LC$ means "light cone". $V$ is the whole volome of the survey. $\delta(\vr;a(r))$ is the matter density perturbation at location $\vr$ with redshift $1/(1+a(r))$. $D$ is the linear growth factor and $D_{\ini} = D(a=a_{\ini})$.\\
We can write the real space matter inhomogeneity field $\delta(\vr;a(r))$ as:
\be 
\delta(\vr;a(r))=\int \frac{d^3 \vbk}{(2\pi)^3}e^{i\vbk\cdot\vr}\delta(\vbk;a(r))
\ee 
Here the wave vector with "bar" means it is a wave vector defined at that specific time $a(r)$. Different from the wave vector $\vk$ in \ec{LC} which means the wave vector is defined in the light cone.\\
Similarly we want to compute the off-diagonal term in order to construct the power spectrum:
\bea 
&&\langle \delta^{(1)}(\vk_s)\delta^{(2)}(\vk_s') \rangle \vs 
&=& \int_V d^{3}\vr\int_{V} d^{3} \vr' \int \frac{d^{3}\vbk}{(2\pi)^3}\int \frac{d^{3}\vbkp}{(2\pi)^3} e^{i(\vbk\cdot\vr+\vbkp\cdot\vr')-i(\vk_s\cdot\vr+\vk_s'\cdot\vr')}\vs
&&\times\big[ \frac{D_{\ini}}{D_{1}(a(r))} \frac{D_{\ini}}{D_{1}(a(r'))} \big]^{2}f_{V}(\vr)f_{V}(\vr') \langle \delta^{(1)}(\vbk;a(r))\delta^{(2)}(\vbkp;a(r')) \rangle \eql{od}
\eea 
We have computed the bracket in \ec{od} before where the result is:
\bea 
&&\langle \delta^{(1)}(\vbk;a(r))\delta^{(2)}(\vbkp;a(r')) \rangle \vs
&=&2\big[\frac{D_{1}(a(r'))}{D_{\ini}} \big]^{2}F_{2}(-\vbk,\vbk+\vbkp)P_{\ini}(\bk)\delta^{(1)}(\vbk+\vbkp;a(r)) \eql{odb}
\eea 
Plugging \ec{odb} into \ec{od}, we can see that the only $\vr'$ dependent integral can be written as:
\be 
\int_{V} d^{3}\vr'f_{V}(\vr')e^{-i(\vbkp-\vk_s')\vr'} = (2\pi)^3\delta(\vbkp-\vk_s')
\ee 
\textbf{is it still true for a finite volome? And whether $\vbkp$ and $\vk_s'$ can show up in the same term?} Thus:
\bea 
&& \langle \delta^{(1)}(\vk_s)\delta^{(2)}(\vk_s') \rangle \vs 
&=&2\int_{V} d^{3}\vr \int \frac{d^{3}\vbk}{(2\pi)^3}[\frac{D_{\ini}}{D_{1}(a(r))} \big]^{2}e^{-i(\vbk-\vk_s)\vr}\delta^{(1)}(\vbk+\vk_s';a(r))\vs
&&\times F_{2}(-\vbk,\vbk+\vk_s')P_{\ini}(\bk)\vs
&=&2\int_{V} d^{3}\vr \int \frac{d^{3}\vbk}{(2\pi)^3}[\frac{D_{\ini}}{D_{1}(a(r))} \big]^{2}e^{-i(\vbk-\vk_s-\vk_s')\vr}\delta^{(1)}(\vbk;a(r)) \vs
&&\times F_2(\vbk,-\vbk+\vk_s')P_{\ini}(|\vbk-\vk_s'|) \vs
&\simeq&2F_{2}(-\vk_s,\vk_s+\vk_s')P_{\ini}(k_s)\delta^{(1)}(\vk_s+\vk_s')
\eea 
The quadratic estimator can be formed as:
\bea 
\hat{\delta}^{(1)}(\vk_l)=A(\vk_l)\int \frac{d^{3}\vk_s}{(2\pi)^3} g(\vk_s,\vk_s')\delta^{\LC}(\vk_s)\delta^{\LC}(\vk_s')
\eea 
with $\vk_s' = \vk_l-\vk_s$.\\ \\ \\
Properties of $f_{V}(\vr)$:
\be 
\int_V d^{3}\vr f_{V}(\vr)e^{-i\vk\cdot\vr}=\int d^{3}\vr \,W(\vr)f_{V}(\vr)e^{-i\vk\cdot\vr} = (2\pi)^3\delta_{V}(\vk)

\ee
where 
\begin{equation}
W(\vr)=\left\{
\begin{aligned}
&1&,\; \vr\in V \\
&0&,\; \vr \notin V
\end{aligned}
\right.
\end{equation}

\end{document}
